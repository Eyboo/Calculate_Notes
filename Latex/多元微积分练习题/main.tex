\documentclass[fontset=windows]{ctexart}%ctexbook,ctexrep
\usepackage{graphicx} % Required for inserting images

% \usepackage[heading=true]{ctex}%一定注意将heading设置为true

\ctexset{
	section={
		%format用于设置章节标题全局格式,作用域为标题和编号
		%字号为小三,字体为黑体,左对齐
		%+号表示在原有格式下附加格式命令
		format+ = \zihao{-3} \songti \raggedright,
		%name用于设置章节编号前后的词语
		%前、后词语用英文状态下,分开
		%如果没有前或后词语可以不填
		name = {第,章{ }},
		%number用于设置章节编号数字输出格式
		%输出section编号为中文
		number = \chinese{section},
		%beforeskip用于设置章节标题前的垂直间距
		%ex为当前字号下字母x的高度
		%基础高度为1.0ex,可以伸展到1.2ex,也可以收缩到0.8ex
		beforeskip = 1.0ex plus 0.2ex minus .2ex,
		%afterskip用于设置章节标题后的垂直间距
		afterskip = 1.0ex plus 0.2ex minus .2ex,
		%aftername用于控制编号和标题之间的格式
		%\hspace用于增加水平间距
		aftername = \hspace{0pt}
	},
	subsection={
		format+ = \zihao{4} \kaishu \raggedright,
		%仅输出subsection编号且为中文
		number = \chinese{subsection},
		name = {第,节{ }},
		beforeskip = 1.0ex plus 0.2ex minus .2ex,
		afterskip = 1.0ex plus 0.2ex minus .2ex,
		aftername = \hspace{0pt}
	},
	subsubsection={
		%设置对齐方式为居中对齐
		format+ = \zihao{-4} \fangsong \centering,
		%仅输出subsubsection编号,格式为阿拉伯数字,打字机字体
		number = \ttfamily\arabic{subsubsection},
		name = {,.},
		beforeskip = 1.0ex plus 0.2ex minus .2ex,
		afterskip = 1.0ex plus 0.2ex minus .2ex,
		aftername = \hspace{0pt}
	}
}

\usepackage{amsfonts,amsmath,amssymb,amsthm}
\usepackage{geometry}
\usepackage{listings,color}
\usepackage[dvipsnames]{xcolor}
\usepackage{verbatim}
\usepackage{enumerate}%编号的宏包

%\geometry{a4paper,scale=0.8}
\geometry{a4paper, left=30 mm, right=25 mm, top=25 mm, bottom= 20 mm}

\title{第三部分}
\author{}
\date{\today}

\begin{document}

\maketitle
\tableofcontents

\newpage

\section{多元函数微分}
\subsection{欧氏空间与多元函数}
\subsubsection{知识概要}
在本节课中,老师总览全局地介绍了多元函数并与之前学过的一元函数相对比,提醒同学们在多元微积分的学习过程中对相关概念的理解可以与一元微积分相对照来帮助理解掌握。这节课程的主要内容是一些基本概念。

\begin{enumerate}
\item 一维欧氏空间是一条直线,在选定原点和单位长度后等同于实数集 $\mathbb{R}^1$,从而有模长(两点之间的距离),大小关系,四则运算和区间的概念。

\item $n$维欧氏空间在确定直角坐标系后等同于$n$元实数集 $\mathbb{R}^n$
$$
\mathbb{R}^n = \{\mathbf{x} = (x_1,x_2,\cdots,x_n)|x_i \in \mathbb{R}^1,i = 1,2,\cdots,n\}
$$
$\mathbb{R}^n$中没有了大小和普通乘除法的概念,但是可以定义出模长(两点之间的距离)、内积、加减法。对于三维时,可以定义出叉乘的概念。这里特意强调两点之间的距离
$$
|\mathbf{a}-\mathbf{b}| = \sqrt{(a_1-b_1)^2+(a_2-b_2)^2+\cdots+(a_n-b_n)^2}
$$

\item 利用两点之间的距离的概念,可以定义邻域,从而引入开集、闭集的概念。记 $n$维欧氏空间中的一点为 $\mathbf{x} =(x_1,x_2,\cdots,x_n) $,以 $\mathbf{x}$为中心的 $\delta$ 邻域是指
$$
U(\mathbf{x},\delta) = \{\mathbf{y} \in \mathbb{R}^n| |\mathbf{x} - \mathbf{y}|<\delta \}
$$
去心邻域是指邻域这个集合内不包括 $\mathbf{x}$这个点
$$
\mathring{U}(\mathbf{x},\delta) = \{\mathbf{y} \in \mathbb{R}^n| |\mathbf{x} - \mathbf{y}|<\delta, \mathbf{y} \neq \mathbf{x} \} = \{\mathbf{y} \in \mathbb{R}^n| 0<|\mathbf{x} - \mathbf{y}|<\delta \}
$$

\item 在邻域这个概念的基础上对欧氏空间中的点集进行分类,某个点 $P\in \mathbb{R}^2$与点集 $E \subset \mathbb{R}^2$的关系分成
\begin{itemize}
    \item 内点:存在 $P$ 的某个邻域,$U(P) \subset E$。也就是说这个点在 $E$ 里面,它邻近还有很多个点属于 $E$,它本身也属于 $E$。
    \item 外点:存在 $P$ 的某个邻域,$U(P) \cap E = \varnothing$。也就是说这个点在 $E$ 外面,它邻近还有很多个点不属于 $E$,它本身也不属于 $E$ 。
    \item 边界点:$P$ 的任一邻域 $U(P) \cap E \ne \varnothing$ 并且 $U(P) \not\subset E$,即任一邻域内既一定含有属于 $E$ 的点,又一定含有不属于 $E$的点。它本身可以属于 $E$ 也可以不属于 $E$ 。
    \item 孤立点:$P$存在一个邻域$U(P)$使得 $U(P) \cap E = P$。它非常孤单。因为周围只有它一个点属于 $E$ ,而其他点都不属于 $E$ 。
    \item 聚点:$\forall \delta >0$,$P$的去心邻域 $\mathring{U}(P, \delta) \cap E \neq \varnothing$,即 $\mathring{U}(P, \delta)$中一定有点属于 $E$。它周围有很邻近的范围都有点属于 $E$,但是它自己可以属于 $E$ ,也可以不属于 $E$ 。
\end{itemize}
这些定义逐字逐句更能领会其精妙。内点一定是聚点,聚点可能是内点可能是边界点,孤立点一定是边界点,边界点可能是孤立点可能是聚点(这三句话可能被考试,但是实际上都更多是文字游戏)。根据点集所属点的特征,对集合进行分类
\begin{itemize}
    \item 有界集:存在某个正数 $r$,使得点集 $E$ 满足 $E \subset U(O,r)$,其中 $O$ 是坐标原点。直观来说就是能从原点画个大圈将点集 $E$包住。
    \item 无界集:不是有界集就是无界集,$\forall r >0 , r \in \mathbb{R}, \exists P \in E, s.t. P \notin U(O,r)$。直观来说就是从原点出发不管画多大个圈都包不住点集  $E$。
    \item 开集:点集 $E$ 中的点全是属于内点。
    \item 闭集:点集的边界 $\partial E \subset E$。
    \item 连通集:点集 $E$ 中的任意两点都可以通过折线连起来,并且折线上的点都属于点集 $E$,即点集 $E$中没有孤立点。
    \item 区域:连通开集,也称为开区域。
    \item 紧集:有界闭集,也称为闭区域。
\end{itemize}

\item 极限行为和描述,数列的极限为
$$
\lim_{n \to \infty} a_n = A \Leftrightarrow \forall \varepsilon>0,\exists N \in\mathbb{N},\mathbf{s.t.}|a_n-A|<\varepsilon,\forall n\geq N
$$
点列的极限为
$$
\lim_{n \to \infty} \mathbf{a}_n = \mathbf{A} \Leftrightarrow \forall \varepsilon>0,\exists N \in\mathbb{N},\mathbf{s.t.}|\mathbf{a}_n- \mathbf{A}|<\varepsilon,\forall n\geq N
$$
两者看上去相似,但是趋近行为上面可以有很大的不同,这个不同主要来源是一维时只有两个方向趋近,而高维时趋近的方式可以很多,极限存在则说明不论以什么方向趋近极限都应该相同。

\item 多元函数,一个多元函数 $f$ 是从 $\mathbb{R}$ 中的一个非空集合 $D$ 到 $\mathbb{R}^1$ 的一个映射
$$
f:D \to \mathbb{R}^1, \mathbf{x} \mapsto f(\mathbf{x})
$$
多元函数变量是多个,但是映射到的结果是一个数。

\item 向量值多元函数,向量值多元函数 $\mathbf{f}$ 是从 $\mathbb{R}^n$ 中的一个非空集合 $D$ 到 $\mathbb{R}^m$中的一个映射
$$
\mathbf{f}:D \to \mathbb{R}^m, \mathbf{x} \mapsto \mathbf{f}(\mathbf{x})
$$
向量值函数是映射的结果是一个向量(具有多个分量),向量值多元函数是变量是多个,映射的结果也是多个的一个函数,每一个分量是一个多元函数。

\item 多元函数的极限,有前面点列的极限基础,对比理解多元函数的极限
$$
\lim_{\mathbf{x} \to \mathbf{x}_0} f(\mathbf{x}) = A
\Leftrightarrow
\forall \varepsilon>0,\exists \delta>0,\text{ s.t. }|f(\mathbf{x})-A|< \varepsilon,\forall 0<|\mathbf{x}-\mathbf{x}_0|<\delta
$$
其几何意义是当点 $\mathbf{x}$ 逼近 $\mathbf{x}_0$ 时,函数值 $f(\mathbf{x})$ 逼近某个确定的值 $A$,类似可以定义多元函数的连续。
$$
\lim_{\mathbf{x} \to \mathbf{x}_0} f(\mathbf{x}) = f(\mathbf{x}_0)
\Leftrightarrow
\forall \varepsilon>0,\exists \delta>0,\textbf{s.t.}f(\mathbf{x})\in U(f(\mathbf{x}_0),\varepsilon),\forall \mathbf{x} \in \mathring{U}(\mathbf{x}_{0},\delta)
$$

\item 重极限和累次极限。多元函数的极限行为都叫重极限。累次极限是指几次有先后的极限行为,比如从 $x$ 方向取极限,然后再从 $y$ 方向取极限。一个关键的定理是如果重极限存在,累次极限也存在,那么它们必定相等。同样的,其逆否命题也很有用,如果累次极限不相等,那么重极限一定不存在
$$
\text{重极限:} \lim_{(x,y) \to (x_0,y_0)} f(x,y) \quad \text{累次极限:} \lim_{y \to y_0}\lim_{x \to x_0} f(x,y) ,\lim_{x \to x_0}\lim_{y \to y_0} f(x,y) 
$$

\item 紧集上的连续函数性质,三条,有界(有最大最小值),一致连续,若定义域连通则满足介值定理。

在上面的复习过程中,以黑体区分了一元和多元,实际书写中和之后的文字中不再以黑体区分,因为一般能通过上下文区分出来,在有可能造成歧义的时候才有专门区分的必要。
\end{enumerate}

\subsubsection{题目示例}
\begin{enumerate}
\item 求函数 $f(x)$ 在原点的极限
$$
f(x,y)=(x^2+y^2)\sin\frac1{x^2+y^2},\quad(x,y)\neq(0,0)
$$

\item 计算
$$
\lim_{(x,y)\to(0,2)}\frac{\sin(xy)}x
$$

\item 证明函数 $f(x)$ 在原点的极限不存在
$$
f(x,y)=\frac{xy}{x^2+y^2},\quad(x,y)\neq(0,0)
$$
\end{enumerate}

\subsubsection{习题参考}
\begin{enumerate}
\item 用数学语言写出下列概念的严格定义:内点、外点、边界点、孤立点、聚点、开集、闭集。

\item 找出集合 $S=\left\{\left(\frac1m,\frac1n\right)\in\mathbb{R}^2|m,n\in\mathbb{N}\right\}$ 的边界点和聚点。

\item 利用极限的定义证明 $m$ 维点列极限的以下性质:
    \begin{enumerate}[(1)]

    \item $\lim _{n\to \infty} \mathbf{a}_n = \mathbf{A}$当且仅当点 $\mathbf{a}_n$的每个位置的分量构成的数列收敛到点 $\mathbf{A}$ 对应分量
    
    \item 若$\lim _{n\to \infty} \mathbf{a}_n = \mathbf{A}$则$\{ \mathbf{a}_n \}$有界
    
    \item $\lim _{n\to \infty} \mathbf{a}_n \cdot \mathbf{b}_n = \lim _{n\to \infty} \mathbf{a}_n \cdot \lim _{n\to \infty} \mathbf{b}_n $
    \end{enumerate}

\end{enumerate}

\subsection{偏导数}
\subsubsection{知识概要}
\begin{enumerate}
\item 采用将多元函数固定方向研究的办法,假设 $f: D \to \mathbb{R}$ 是一个 $n$ 元函数,$x_0 \in D$是一个定点,选定一个单位向量 $v \in \mathbb{R}^n$,让自变量沿着 $v$方向变动,得到一个一元函数
$$
\phi _v(t) = f(x_0 + t v) 
$$
容易定义这个一元函数的导数(以及左导数、右导数)
$$
\frac{\mathrm{d} \phi_ v}{\mathrm{d} t}(0)=\lim_{t\to 0}\frac{f(x+t v)-f(x)}{t}
$$

\item 如果将上面说的单位向量选在某个坐标轴上,就得到多元函数 $f$ 在关于 $x_i$ 方向的偏导数定义
$$
\partial_i f(x) = \lim_{t\to 0}\frac{f(x+t e_i)-f(x)}{t}
$$
其中,$e_i = (0,\cdots,1,\cdots,0)$是 $x_i$ 方向上的单位向量。

\item 在普遍情况下是方向导数的定义,设 $v\in \mathbb{R}^n$的一个单位向量,多元函数 $f:D\to \mathbb{R}$在点 $x$处关于 $v$方向的方向导数定义为
$$
\frac{\partial f}{\partial\nu}(x)=\lim_{t\to0^+}\frac{f(x+t\nu)-f(x)}t
$$
注意方向导数中极限是一个右极限(从右边靠近过来)。
\end{enumerate}

\subsubsection{题目示例}
\begin{enumerate}
\item 曲面$z=f(x,y)=\sqrt{9-2x^2-y^2}$与平面$y=1$交于一条曲线$\gamma.$试求曲线
$\gamma$在$(\sqrt2,1,2)$点的切线.

\item 求函数 $f(x)$ 在 $(0,0)$的偏导数
$$
f(x,y)=\begin{cases}\frac{xy}{x^2+y^2},&(x,y)\neq(0,0)\\0,&(x,y)=(0,0) \end{cases}
$$

\end{enumerate}
\subsubsection{习题参考}
\begin{enumerate}
\item 证明函数
$$
u(x)=\frac1{|x|}:\mathbb{R}^3\setminus\{0\}\to\mathbb{R}
$$
满足方程
$$
\Delta u=\frac{\partial^2u}{\partial x^2}+\frac{\partial^2u}{\partial y^2}+\frac{\partial^2u}{\partial z^2}=0.
$$

\item 设 $e_i$是沿 $x_i$ 方向的单位向量,证明 $f$ 在 $x$ 点对 $x_i$ 的偏导数存在的充要条件是: $f$ 在 $x$ 点沿 $e_i$ 和 $-e_i$ 的方向导数存在且互为相反数。

\item (教材第71页习题9-2第1题) 求下列函数的偏导数
\begin{align*}
(1). z &= x^3y-y^3x  & (2). s &= \frac{u^2+v^2}{u v} \\
(3). z &= \sqrt{\ln(x y)} &   (4). z &= \sin(x y)  + \cos ^2(x y) \\
(5). z &= \ln(\tan(\frac{x}{y})) & (6). z &= (1+x y)^y \\
(7). u &= x^{\frac{y}{z}} & 8. u &= \arctan (x-y)^z 
\end{align*}

\item (教材第71页习题9-2第5题)曲线 $L$ 在点 $(2,4,5)$ 处的对于 $x$ 轴的倾角为多少?
$$
L:\left\{\begin{matrix}
z = \frac{x^2+y^2}{4}\\
y = 4
\end{matrix}\right.
$$

\item (教材第71页习题9-2第7题)设 $f(x,y,z) = x y^2+y z^2+z x^2$,求 $f_{xx}(0,0,1)$, $f_{xz}(1,0,2)$, $f_{yz}(0,-1,0)$, $f_{zzx}(2,0,1)$
\end{enumerate}

\subsection{全微分}
\subsubsection{知识概要}
一元微积分的微分几何意义非常明确并且非常有用,即用直线来逼近曲线,微分 $\mathrm{d} y$指的就是当 $x$ 变化 $\Delta x$时,用来逼近的线性函数变化的量。可微的定义是,当 $\Delta x  \to 0$时,函数的增量满足
$$
\Delta y = f(x_0 + \Delta x) - f(x_0) = A \Delta x + o(\Delta x)
$$
其中 $A$ 是不依赖于 $\Delta x$的数($A$这个数存在),称为 $f(x)$在 $x_0$处可微。称 $dy = A \Delta x$为函数 $f(x)$在 $x_0$处的微分,函数在该点的线性逼近可以写为
$$
y - y_0 = A(x-x_0)
$$
这就是曲线在该点的切线方程,容易证明可微是可导的充分必要条件。对于多元微积分,从几何直觉上讲,我们期望能类比切线找到切平面,如用直线对曲线做线性近似,也可以用平面对曲面做线性近似。

多元函数的微分称为全微分,定义为当 $\Delta x = (\Delta x _1, \Delta x_2, \cdots, \Delta x_n) \to 0$时,函数 $y = f(x)$满足
$$
\Delta y = f(x_0 + \Delta x) - f(x_0) = A \cdot \Delta x + o(|\Delta x|)
$$
这时,$A$为一个数组(行向量$A = (a_1, a_2, \cdots, a_n)$)不依赖于 $\Delta x$(这样一个数组存在),则称 $f$在 $x_0$点可微,其中 $\Delta y$的线性部分被称为 $f$的全微分
$$ 
\mathrm{d} y = A \cdot \Delta x = \begin{bmatrix}
a_1  & a_2 & \cdots & a_n
\end{bmatrix}
\cdot
\begin{bmatrix}
\Delta x_1 \\
\Delta x_2 \\
\vdots \\
\Delta x_n
\end{bmatrix}
= a_1 \Delta x_1 + a_2 \Delta x_2 + \cdots + a_n \Delta x_n
$$
即可对照一维切线方程写出这里的切平面方程
$$
y - y_0 = A \cdot \Delta x \Leftrightarrow y - y_0 = a_1 (x_1-x_{10}) + a_2 (x_2 - x_{20}) + \cdots + a_n (x-x_{n0})
$$

全微分与偏导由一个定理关联起来:若多元函数 $f:D ^n \to \mathbb{R}$在点 $x_0 \in D$可微,则 $f$ 在点 $x_0$连续,$f$在点 $x_0$的所有偏导数都存在,且有
$$
dy = \partial _1 f(x_0) \mathrm{d} x_1 + \partial _2 f(x_0) \mathrm{d} x_2  + \cdots + \partial _n f(x_0) \mathrm{d} x_n
$$
即 $A = (\partial _1 f(x_0)  + \partial _2 f(x_0)   + \cdots + \partial _n f(x_0) )$。这个证明与一元的微分与导数的证明过程类似,可以由此把玩一下导数与微分之间的关系。这个定理也就是说可微的条件明显比可偏导更苛刻的多,要函数在某一点所有的偏导数都存在并且函数在该点连续的时候函数在该点才可微。

在上面的讨论过程中,我们已经有感觉求微分时的那个不受 $\Delta x$影响的 $A$ 貌似很重要,实际上也确实如此,它称为梯度,定义如下:若 $f$ 在 $x$ 处可微,则 $n$维向量
$$
\nabla f (x) = (\partial _1 f(x)  + \partial _2 f(x)   + \cdots + \partial _n f(x))
$$
称为函数 $f$ 在 $x$ 处的梯度,对于一维来说,就像是导数一样,导数数字有大小(正负),这里梯度是一个向量(有大小有方向的量)。梯度最重要的一个性质是,对于任意单位向量 $v$,函数$f$ 在 $v$方向上的方向导数为
$$
\frac{\partial f}{\partial v} = \nabla f \cdot v
$$
又因为 $\nabla f \cdot v = |\nabla f| \cos \theta \ge |\nabla f| $,其中 $\theta$是 $\nabla f$与单位向量 $v$ 之间的夹角,这里就说明,梯度的模长是方向导数中最大的那个数,也就是说沿着梯度的方向,函数变化最快!

在上面我们对于多元函数求了微分,如果是向量值多元函数呢?能够类比推导其微分应该是一个向量等于一个矩阵乘以另外一个向量,也就是将上面的行向量(梯度)变成了矩阵,这个矩阵称为雅可比矩阵,可以记作 $\mathrm{D} F(x)$
$$
DF(x):=\left(\begin{array}{cccc}\partial_{1}f_{1}(x)&\partial_{2}f_{1}(x)&\cdots&\partial_{n}f_{1}(x)\\
\partial_{1}f_{2}(x)&\partial_{2}f_{2}(x)&\cdots&\partial_{n}f_{2}(x)\\
\vdots&\vdots&\vdots&\vdots\\
\partial_{1}f_{m}(x)&\partial_{2}f_{m}(x)&\cdots&\partial_{n}f_{m}(x)
\end{array}\right)_{m\times n}
$$
稍微回顾一下线性代数的内容,会发现这里单值一元到多元,再到向量值多元函数他们的微分形式是多么的类似,这应该也反映着他们的本质也是类似的。
\subsubsection{题目示例}
\begin{enumerate}
    \item 求函数 $u = x + \sin \frac{y}{2} + \mathrm{e} ^{yz}$的全微分。
    \item 求函数 $z=\mathrm{e}^{x y}$在点 $(2,1)$的切平面。
    \item 函数$f = \left\{\begin{matrix}
    |x|^2 \sin \frac{1}{|x|^2}& x \ne 0; \\
    0&x=0
    \end{matrix}\right. $在原点是否可微?
\end{enumerate}

\subsubsection{习题参考}
\begin{enumerate}
    \item 证明函数 $u(x)=|x|:\mathbb{R} ^n \to \mathbb{R}$满足
    $$
    |\nabla u(x) | = 1, \forall x \in \mathbb{R}^n
    $$
    
    \item (教材第77页习题9-3第1题)求下列函数的全微分
    \begin{align*}
    (1). z & = x y + \frac{x}{y}; &  (2). z &= \mathrm{e}^{\frac{y}{x}}; \\
    (3). z &= \frac{y}{x^2 + y^2}; & (4). u & = x^{yz}
    \end{align*}
    
    \item (教材第77页习题9-3第2题)求函数 $z = \ln (1+x^2 + y^2)$当 $x=1,y=2$时的全微分。
    
    \item (教材第77页习题9-3第3题)求函数 $z = \frac{y}{x}$当 $x=2,y=1, \Delta x = 0.1, \Delta y = -0.2$时的全增量和全微分。
\end{enumerate}

\subsection{链式法则}
\subsubsection{知识概要}
首先介绍了复合函数的概念,多元函数的复合函数的概念和一元函数是类似的,都是映射两次。然后介绍了多元复合函数的求导法则,即链式法则。
\begin{enumerate}
    \item 设 $f: \mathbf{D}^n \to \mathbb{R}^m$和 $g: \mathbf{\Omega} ^m \to \mathbb{R} ^l $是两个(可以是向量值)的函数,如果 $R(f) \subset \mathbf{D}(g) $就可以定义复合函数
    
    $$
    g \circ f : \mathbf{D} \to \mathbb{R} ^ l , x \mapsto g(f(x))
    $$
    这里注意数学叙述中对定义域和值域严格的描述。
    
    一个容易理解的定理保证了复合函数的连续性:如果 $f$ 在 $x$ 点连续,且 $g$ 在点 $y=f(x)$ 处连续,则复合函数 $g \circ f $在 $x$ 点连续。这个定理很符合直觉,也很容易得到证明(严格的证明过程略难,但是证明过程或者说方向是容易思考的),所以比较容易理解。 
    
    \item  一元函数的链式法则,
    $$
    \frac{dz}{dx}=\frac{dz}{dy}\frac{dy}{dx}
    $$
    等价于, $ (g\circ f)^\prime(x)=g^\prime(y)f^\prime(x)$,由此也可以写出一阶微分的形式不变性
    $$
    dz=\frac{dz}{dy}dy=\frac{dz}{dx}dx
    $$
    \item 多元函数的链式法则,设多元函数 $y = f(x)$和 $z = g(y)$ 均可微,定义雅可比矩阵
    $$
    \frac{Dz}{Dx}=\left(\frac{\partial z_i}{\partial x_j}\right)_{l\times n},\frac{Dz}{Dy}=\left(\frac{\partial z_i}{\partial y_k}\right)_{l\times m},\frac{Dy}{Dx}=\left(\frac{\partial y_k}{\partial x_j}\right)_{m\times n}
    $$
    有
    $$
    \left.\left(\begin{array}{ccc}\frac{\partial z_1}{\partial x_1}&\cdots&\frac{\partial z_1}{\partial x_n}\\\vdots&\vdots&\vdots\\\frac{\partial z_l}{\partial x_1}&\cdots&\frac{\partial z_l}{\partial x_n}\end{array}\right.\right)=\left(\begin{array}{ccc}\frac{\partial z_1}{\partial y_1}&\cdots&\frac{\partial z_1}{\partial y_m}\\\vdots&\vdots&\vdots\\\frac{\partial z_l}{\partial y_1}&\cdots&\frac{\partial z_l}{\partial y_m}\end{array}\right)\cdot\left(\begin{array}{ccc}\frac{\partial y_1}{\partial x_1}&\cdots&\frac{\partial y_1}{\partial x_n}\\\vdots&\vdots&\vdots\\\frac{\partial y_m}{\partial x_1}&\cdots&\frac{\partial y_m}{\partial x_n}\end{array}\right)
    $$
    从一阶微分的形式不变性看这个式子,就是把数字换成了矩阵,根据矩阵乘法,对于任意的 $i = 1,2,3, \cdots, l$和 $j = 1,2,3,\cdots,n$,都有
    $$
    \frac{\partial z_i}{\partial x_j}=\frac{\partial z_i}{\partial y_1}\frac{\partial y_1}{\partial x_j}+\cdots+\frac{\partial z_i}{\partial y_m}\frac{\partial y_m}{\partial x_j}=\sum_{k=1}^m\frac{\partial z_i}{\partial y_k}\frac{\partial y_k}{\partial x_j}
    $$
    用累加符号表示时显得凝练,用矩阵表示时,更加凝练并且有好像抓住了更深层次规律的感觉,使用合适的符号来进行表达,思维和展示都会显得更加唯美。
    $$
    \frac{Dz}{Dx}=\frac{Dz}{Dy}\cdot\frac{Dy}{Dx}
    $$
    这等价于
    $$
    D (g \circ f )(x) = Dg(y) \cdot Df(x) \Leftrightarrow  dz=Dg(y)dy=Dg(y)\cdot Df(x)dx
    $$
\end{enumerate}

\subsubsection{题目示例}
\begin{enumerate}
    \item 设 $z = \mathrm{e} ^u \sin v$且 $u = xy$, $v = x+y$.求 $\frac{\partial z}{\partial x}, \frac{\partial z}{\partial y}.$
    
    \item 设 $u = f(x,y,z) = \mathrm{e} ^{x^2 + y^2 + z^2}$且 $z = x^2 \sin y$.求 $\frac{\partial u}{\partial x}$, $\frac{\partial u}{\partial y}$.
    
    \item 设 $z = f(u,v,t) = u v+ \sin t$且 $u = \mathrm{e}^t$,$v = \cos t$,求 $\frac{\mathrm{d} z}{ \mathrm{d} t}$
    
    \item 设 $w = f(x+y+z, xyz)$ 且 $f \in C^2$.求 $\frac{\partial w}{\partial x}$, $\frac{\partial ^2 w}{\partial x \partial z}$
    
    \item 设 $u=f(x,y) \in  C^2$. 在极坐标中计算
    $$
    (1) |\nabla u|^2 = \left( \frac{\partial u}{\partial x} \right)^2 + \left( \frac{\partial u}{\partial y} \right)^2 ; \quad (2) \Delta u = \frac{\partial ^2 u}{\partial x^2} + \frac{\partial ^2 u}{\partial y^2}
    $$
\end{enumerate}
\subsubsection{习题参考}
\begin{enumerate}
    \item (教材第85页习题9-4第6题) 设 $ u = \frac{\mathrm{e}^{ax} (y-z)}{a^2+1}$, 而 $y=a \sin x$, $z = \cos x$ , 求 $\frac{\mathrm{d} u}{ \mathrm{d} x}$

    \item (教材第85页习题9-4第8题) 求下列函数的一阶偏导数(其中 $f$ 具有一阶连续偏导数)
    \begin{align*}
        &(1) u = f(x^2-y^2, \mathrm{e} ^{xy}) ; & (2) u = f\left(\frac{x}{y},\frac{y}{z} \right); \\
     &(3) u = f(x,xy,xyz).
    \end{align*}

    \item (教材第85页习题9-4第10 题) 设 $z = \frac{y}{f(x^2 - y^2)}$, 其中 $f(u)$ 为可导函数,验证
    $$
    \frac{1}{x} \frac{\partial z}{\partial x} + \frac{1}{y}\frac{\partial z}{\partial y} = \frac{z}{y^2}
    $$

    \item (教材第85页习题9-4第11题) 设 $z = f(x^2+y^2)$, 其中 $f$ 具有二阶导数,求 $\frac{\partial ^2 z}{\partial x^2}$ , $\frac{\partial ^2 z}{\partial x \partial y}$, $\frac{\partial ^2 z}{\partial y^2}$
\end{enumerate}

\subsection{隐函数定理与反函数定理}
\subsubsection{知识概要}
在一元微积分时候学过反函数的求导法则,反函数的倒数与原函数的导数乘积为1,反函数的导数是原函数导数的倒数,对于多元函数,是不是将概念1扩展成单位阵,将倒数扩展成逆就可以了呢?

\begin{enumerate}
    \item 之所以要了解隐函数求导,是因为在一些情况下隐函数方程是容易得到的,但是显式方程不容易得到,并且从局部上看,很多函数除去一些特殊的点之外都可以将一些变量表示成另外一些变量的函数。首先看方程中只有两个变量的,如果一个二元函数,$F(x,y): D^2 \to \mathbb{R}$ 满足
    \begin{enumerate}[(1)]
        \item 在点 $P=(x_0, y_0) \in D $ 的一个邻域上 $ F \in  C^2 $
        \item $F(x_0, y_0) = 0$
        \item $F_y(x_0, y_0) \neq 0$
    \end{enumerate}
    则
    \begin{enumerate}[(1)]
        \item 存在 $P$ 点的一个邻域,使得方程 $F(x,y) =0$唯一决定了一个隐函数 $y = f(x)$, 满足
        $$ F(x, f(x)) =0 $$
        \item 函数 $y = f(x)$ 在此邻域上可微
        \item $f$ 的导数满足
        $$ \frac{\mathrm{d} y}{\mathrm{d} x } = - \frac{F _x}{F _y} $$
    \end{enumerate}

    \item 如果方程中有多个变量,可以将其中一个看成其他许多个变量的函数,设多元函数 $F(x,y): D^{n+1} \to \mathbb{R} ^1$ (其中 $x = (x_1,x_2,\cdots, x_n)$是 $n$ 维变量)满足
    \begin{enumerate}[(1)]
        \item 在点 $P=(x_0 , y_0 ) \in D$ 的一个邻域上 $F \in C^1$
        \item $F(x_0, y_0) = 0$
        \item $F_y (x_0, y_0) \neq 0$
    \end{enumerate}
    则
    \begin{enumerate}[(1)]
        \item 存在 $P$ 点的一个邻域,使得方程 $F(x,y)=0$唯一决定了一个隐函数 $y = f(x)$,满足 
        $$ F(x,f(x)) = 0 $$
        \item 函数 $y = f(x)$在此邻域上可微
        \item $f$ 的导数满足
        $$ \frac{\partial y}{ \partial x_i} = - \frac{F_{x_i}}{F_y}, \forall i = 1,2,\cdots , n $$
    \end{enumerate}

    \item 如果有很多变量和很多方程呢?一个方程使得其中一个变量看成另外其他变量的函数,$m$ 个方程可以使得其中 $m$ 个变量看成其他变量的函数。设向量值多元函数 $F(x,y): D^{n+m} \to \mathbb{R} ^ {m}$ (其中$x=(x_1,x_2, \cdots, x_n)$是 $n$ 维变量, $y= (y_1,y_2,\cdots,y_n)$是 $m$ 维变量),满足
    \begin{enumerate}[(1)]
        \item 在点 $P=(x_0 , y_0 ) \in D$ 的一个邻域上 $F \in C^1$
        \item $F(x_0, y_0) = 0$
        \item $ \frac{\partial F}{ \partial y}(x_0, y_0) = \mathrm{det} \frac{D F}{D y}(x_0, y_0) \neq 0$
    \end{enumerate}
    则
    \begin{enumerate}[(1)]
        \item 存在 $P$ 点的一个邻域,使得方程 $F(x,y)=0$唯一决定了一个隐函数 $y = f(x)$,满足 
        $$ F(x,f(x)) = 0 $$
        \item 函数 $y = f(x)$在此邻域上可微
        \item $f$ 的导数满足
        $$ \frac{D y}{ D x} = - \left( \frac{D F}{D y} \right)^{-1} \frac{D F}{D x}$$
    \end{enumerate}
    
    \item 这三个结论均可以这样类似地进行推导,隐函数 $y = f(x)$,满足 $F(x,f(x))=0$ 并可微,方程两边同时对 $x$ 求导,得到
    $$
    \frac{DF}{Dx}+\frac{DF}{Dy}\frac{Dy}{Dx}=0 
    $$
    这等价于线性方程组
    $$
    \left(\begin{array}{ccc}\frac{\partial F_1}{\partial y_1}&\cdots&\frac{\partial F_1}{\partial y_m}\\\vdots&\ddots&\vdots\\\frac{\partial F_m}{\partial y_1}&\cdots&\frac{\partial F_m}{\partial y_m}\end{array}\right)\cdot\left(\begin{array}{c}\frac{\partial y_1}{\partial x_j}\\\cdots\\\frac{\partial y_m}{\partial x_j}\end{array}\right)=-\left(\begin{array}{c}\frac{\partial F_1}{\partial x_j}\\\cdots\\\frac{\partial F_m}{\partial x_j}\end{array}\right)
    $$
    由克莱姆法则,可以解出
    $$
    \frac{\partial y_i}{\partial x_j}=-\frac{\partial(F_1,\cdots,F_i,\cdots,F_m)}{\partial(y_1,\cdots,x_j,\cdots,y_m)}/\frac{\partial(F_1,\cdots,F_i,\cdots,F_m)}{\partial(y_1,\cdots,y_i,\cdots,y_m)}
    $$
    这个结论可以巧记,分母都是一样的(系数矩阵),分子的每次只改变一列,改变的那一列与对哪个变量求导有关。

    \item 反函数定理,假设向量值多元函数 $y=f(x): \mathbb{R} ^n \to \mathbb{R} ^n$具有连续的偏导数,并且
    $$
    \frac{\partial y}{\partial x} (x_0) = \mathrm{det} \frac{D y}{D x}(x_0) \neq 0
    $$
    那么在点 $(x_0, y_0)$附近存在一个邻域,以及定义在此邻域上的一个反函数 $x = f^{-1}(y)$,满足
    $$
    \frac{Dx}{Dy}=D(f^{-1})=(Df)^{-1}=\left(\frac{Dy}{Dx}\right)^{-1}
    $$
    隐函数定理和反函数定理是等价的。可以相互推导的
\end{enumerate}

\subsubsection{题目示例}
\begin{enumerate}
    \item 设 $x^2 + y^2 + z^2 -4z =0$,计算 $\frac{\partial z}{\partial y}$ 和 $\frac{\partial ^2 z}{\partial x ^2}$
    
    \item 设 $\begin{cases} x u -yv=0,\\ yu + x v= 1 \end{cases}$,计算 $\frac{\partial u}{\partial x}, \frac{\partial u}{\partial y}, \frac{\partial u}{\partial x}, \frac{\partial u}{\partial y}$
\end{enumerate}

\subsubsection{习题参考}
\begin{enumerate}
    \item (教材第92页习题9-5第7题)设 $\Phi(u,v)$具有连续偏导数,证明由方程 $\Phi (cx-az , cy - bz) =0 $所确定的函数 $z = f(x,y)$ 满足 $a \frac{\partial z }{\partial x} + b \frac{\partial z}{\partial y}=c$.

    \item (教材第92页习题9-5第8题)设 $z -3 x y z = a^3$,求 $\frac{\partial ^2 z}{\partial x \partial y}$.

    \item (教材第92页习题9-5第10题) 求下列方程组所确定的函数的导数或偏导数
    \begin{enumerate}[(1)]
        \item 设$\begin{cases} z = x^2 + y^2,\\ x^2 + 2 y^2 + 3z^2 = 20, \end{cases}$ 求 $\frac{\mathrm{d} y }{\mathrm{d}x}$, $\frac{\mathrm{d} z}{\mathrm{d} x}$;
        \item 设 $\begin{cases}
            x+ y + z = 0, \\
            x^2 + y^2 + z^2 =1,
        \end{cases}$
        求 $\frac{\mathrm{d} x}{\mathrm{d} z}$, $\frac{\mathrm{d} y }{\mathrm{d}z}$;
        \item 设 $\begin{cases}
            u = f(uv,v+y), \\
            v = g(u-x,v^2 y),
        \end{cases}$
        其中 $f,g$具有一阶连续偏导数,求 $\frac{\partial u }{\partial x}$, $\frac{\partial v }{\partial x}$;
        \item 设 $\begin{cases}
            x = \mathrm{e}^u + u \sin v , \\
            y = e^u - u \cos v,
        \end{cases}
        $求,$\frac{\partial u}{\partial x}, \frac{\partial u}{\partial y}, \frac{\partial v}{\partial x}, \frac{\partial v}{\partial y}$.
    \end{enumerate}
\end{enumerate}

\subsection{几何中的应用}
\subsubsection{知识概要}
\begin{enumerate}
   \item 直线的确定。给定3维空间中的一个点和一个非零向量就可以确定一条直线,设该点为 $P_0 = (x_0, y_0, z_0)$该向量为 $\vec{v} = (a,b,c) \neq 0$,过 $P_0$ 并与 $\vec{v}$平行的直线上的任意一点 $P$ 必然满足 $ \vec{ P P_0} || \vec{v} \Leftrightarrow \vec{P P_0} = t \vec{v} , t \in \mathbb{R} $,由此可以得到直线的(参数)方程
    $$
    \begin{cases}
        x(t) = a t + x_0 \\
        y(t) = b t + y_0 \\
        z(t) = c t + z_0
    \end{cases}
    $$
    其中 $t \in \mathbb{R} ^1$ 为参数,或者写成
    $$
    \frac{x-x_0}{a}=\frac{y-y_0}{b}=\frac{z-z_0}{c}
    $$
    写成这样的形式时当分母为 $0$ 时,要求分子也为 $0$. 经过这里的梳理,看到一个直线的参数方程或者对称式时应该可以从中读出该直线的方向向量。

    \item 确定3维欧氏空间中的一个平面,可以由一个点 $P_0 = (x_0, y_0, z_0)$和两个线性无关的向量 $ \vec{v}_1 = (a_1,b_1,c_1), \vec{v}_2 = (a_2,b_2,c_2) \in \mathbb{R}^3$确定,经过该点 $P_0$并且平行于 $\vec{v}_1, \vec{v}_2$的平面上任意一点必然满足
    $$
    \vec{P P_0} = \lambda \vec{v}_1 + \mu \vec{v}_2, \quad \lambda, \mu \in \mathbb{R}
    $$
    由此得到平面的(参数)方程
    $$
    \begin{cases}
        x(\lambda, \mu) = a_1 \lambda + a_2 \mu + x_0 \\
        y(\lambda, \mu) = b_1 \lambda + b_2 \mu + y_0 \\
        z(\lambda, \mu) = c_1 \lambda + c_2 \mu + z_0
    \end{cases}
    $$
    其中 $\lambda, \mu \in \mathrm{R} ^1 $为参数。指的指出的是向量 $\vec{v}_1$和 $\vec{v}_2$ 线性无关等价于 $\vec{v}_1 \times \vec{v}_2 \neq 0$.

    \item 确定一个平面还可以由一个点 $P_0 = (x_0, y_0, z_0)$和一个非零的向量 $ \vec{n} = (a,b,c) \neq 0 $确定,过点 $P_0$并且垂直于向量 $\vec{n}$ 的平面上任意一点必然满足 
    $$
    \vec{ P P_0} \bot \vec{n} \Leftrightarrow \vec{P P_0} \cdot \vec{n} = 0
    $$
    由此得到平面的方程(点法式)
    $$
    a(x-x_0) + b(y - y_0) + c (z - z_0) = 0
    $$
    要能根据点和法向量写出平面的点法式方程,也要能从点法式中分析出平面的法向量。这里的法向量与前一种对平面的表达方式的联系在于 $\vec{n} = \vec{v}_1 \times \vec{v}_2 $

    \item 容易知道,一个方程控制住一个变量(减少一个自由变量),对于一根曲线,如果是参数方程,肯定只有一个参数。空间中任意曲线的参数方程假设为
    $ x = x(t), y=y(t), z = z(t)$,要求参数方程是一阶可导的,记 $\vec{v}(t) = (x'(t), y'(t), z'(t)) \neq \vec{0}$,$\vec{v}(t)$就是该曲线的切向量,从而可以写出曲线上某点的切线方程和发平面方程。

    \item 3维欧氏空间中的曲面一般方程可以写为
    $$
    \Sigma: F(x,y,z) =0
    $$
    一共三个变量,一个方程减少了一个自由变量,剩下两个自由变量符合平面需要的两个自由度。该方程 $F$是一阶可微的,$\nabla F = (F_x, F_y, F_z) \neq 0$,曲面上一点 $P_0 \in \Sigma $的一个法向量为 $\vec{n} = \nabla F(P_0)$,有了法向量就可以写出过该点的切平面方程和法线方程。
    
    6. 对于曲面 $\Sigma : F(x,y,z)=0$,为什么 $\nabla F$是曲面的一个法向量呢?从数学证明上,可以设 $\phi:(-1,1) \to \mathbb{R}^3$是曲面 $\Sigma$ 上任意一条过 $P_0$ 点的曲线,其中 $\phi(0) = P_0$,那么
    $$
    \forall t \in (-1,1), \phi (t) \in \Sigma \Leftrightarrow F(\phi(t)) = 0
    $$
    对上面这个方程在 $t=0$处求导,得到
    $$
    \nabla F (P_0) \cdot \phi '(0) =0
    $$
    这也就说明,向量 $\nabla F(P_0)$ 垂直于曲线 $\phi$ 在 $P_0$点的切向量。所以 $\nabla F$是曲面 $\Sigma$的一个法向量。有没有其他更加直观的理解方式呢?

    \item 知道求取曲面法向量的方法,如果一个曲面以一般形式告诉我们即曲面 $\Sigma : F(x,y,z)=0$,那么在某一点 $P_0 = (x_0,y_0,z_0)$处,曲面的切平面为
    $$
    (x-x_0,y-y_0,z-z_0) \cdot (F_x, F_y, F_z) = 0
    $$
    过该点的法线方程为
    $$
    \frac{x-x_0}{F_x} = \frac{y-y_0}{F_y} = \frac{z-z_0}{F_z}
    $$
    如果曲面给出的是显示方程如 $z = f(x,y)$,那么将其改写成一般形式 $F(x,y,z) = f(x,y)-z = 0$,然后就可以使用上面推出的一般形式,此时 $\nabla F = (f_x,f_y,-1)$。值得注意的是,如果将曲面 $F(x,y,z) = f(x,y)-z=0$作成等高线图,那么 $\nabla f = (f_x, f_y)$是垂直于等高线的(变化最快的方向)。

    \item 如果曲面是由参数方程给出的
    $$
    \Sigma:\begin{cases}x=x(u,v)\\y=y(u,v)\\z=z(u,v)&\end{cases},(u,v)\in D
    $$
    假设 $\frac{\partial(x,y)}{\partial(u,v)}\neq 0$,那么由隐函数定理可知,存在局部隐函数使得
    $$
    u=u(x,y),\nu=\nu(x,y)
    $$
    于是依然可以得到曲面局部的显示方程
    $$
    \Sigma:z=z(u(x,y),\nu(x,y))
    $$
    从而得到法向量
    $$
    \vec{n}=(z_x,z_y,-1)\quad \parallel \left(\frac{\partial(y,z)}{\partial(u,v)},\frac{\partial(z,x)}{\partial(u,v)},\frac{\partial(x,y)}{\partial(u,v)}\right)
    $$
    与法向量平行的这个向量是由隐函数求导退出来的,可以当作练习隐函数的一个机会。

    \item 如果对于曲线的一般方程(曲线的一般方程即用两个方程控制两个变量,使得只有一个自变量,可以理解成两个曲面的交线)
    $$
    \gamma:\begin{cases}\Sigma_1:F(x,y,z)=0\\\Sigma_2:G(x,y,z)=0&\end{cases}
    $$
    如何求这个曲线的切线呢?有了上面使用隐函数的经验,这里显然三个变量两个方程,可以将其中两个变量看成是另外一个变量的函数,例如看成 $y = y(x), z=z(x)$,这样曲线的一般方程就由隐函数来确定
    $$
    \gamma:\begin{cases}\Sigma_1:F(x,y(x),z(x))=0\\\Sigma_2:G(x,y(x),z(x))=0&\end{cases}
    $$
    曲线的切向量就可以写成 $(1,y'(x),z'(x))$,这个切向量中的导数可以用隐函数定理求出来。另外一种求取方式是这样想,曲线由两个平面相交而成,那么这个曲线在曲面 $\Sigma _1$上,它的切向量是垂直于曲面 $\Sigma _1$的法向量 $\nabla F$ 的,同理也垂直于 $\Sigma _2$的法向量 $\nabla G$的,这两个法向量根据已知的方程都容易求出,他们的叉积方程和曲线切向量的方向相同(叉积同时垂直于两个做叉的向量的)
    $$
    \vec{t}=\vec{n}_{1}\times\vec{n}_{2}=\nabla F\times\nabla G\mathrm{~}(\neq\vec{0}) \left.=\left|\begin{array}{ccc}\vec{i}&\vec{j}&\vec{k}\\F_x&F_y&F_z\\G_x&G_y&G_z\end{array}\right.\right|=\left(\frac{\partial(F,G)}{\partial(y,z)},\frac{\partial(F,G)}{\partial(z,x)},\frac{\partial(F,G)}{\partial(x,y)}\right)
    $$
    曲面参数方程的法向量表达式,一般曲线方程切向量的表达式最终的结果是有一定规律从而容易记忆的,可以记住。
\end{enumerate}

\subsubsection{题目示例}
\begin{enumerate}
    \item 求下面曲线在指定点的切线和法平面给
    \begin{enumerate}[(1)]
        \item $x = t, y = t^2 , z = t^3$在点 (1,1,1)处。
        \item $\begin{cases}
            x^2 + y^2 +z^2 = 6 \\
            x + y + z =0 
        \end{cases}$
        在点 $(1,-2,1)$处。
    \end{enumerate}

    \item 求下面平面在指定点的切平面和法线
    \begin{enumerate}[(1)]
        \item $x^2 + y^2 + z^2 = 14$在点 $(1,2,3)$处。
        \item $z = x^2 + y^2 -1$在点 $(2,1,4)$处。
    \end{enumerate}
\end{enumerate}

\subsubsection{习题参考}
\begin{enumerate}
    \item (教材第102页习题9-6第6题) 求曲线
    $\begin{cases}
            x^2 + y^2 + z^2 - 3x = 0, \\
            2 x + 3y + 5z - 4 = 0
        \end{cases}
    $在点 $(1,1,1)$处的切线及法平面方程。

    \item (教材第92页习题9-5第9题) 求曲面 $a x^2 + b y^2 + c z^2 = 1$在点 $(x_0, y_0, z_0)$处的切平面及法线方程。

    \item (教材第92页习题9-5第10题) 求椭球面 $x^2 + 2 y^2 + z^2 =1$上平行于平面 $x-y+2z=0$的切平面方程。

    \item (教材第92页习题9-5第11题) 求旋转椭球面 $3x^2 + y^2  + z^2 =16 $上点 $(-1, -2 , 3)$处的切平面与 $ xOy $ 面的夹角的余弦。
\end{enumerate}

\subsection{极值问题}
\subsubsection{知识概要}
\begin{enumerate}
    \item 首先要清楚极值(最大值、最小值、局部极大值、局部极小值)和极值点(最大值点、最小值点、局部极大值点、局部极小值点)以及驻点的定义,奇异点(不可导的点)称为函数的一个临界点。也要会用数学语言表达这些定义,现在举例局部极小值点的定义,其余应该要能类推写出来:给定函数 $f : D \to \mathbb{R}$和一个点 $P_0 \in D$, 如果 $P_0$ 点附近存在一个邻域 $U$,使得
    $$
    f(P) \geq f(P_0), \quad \forall P \in U \cap D
    $$
    则称 $f(P_0)$是函数 $f$ 的局部极小值,称 $P_0$是局部极小值点。

    \item 一元函数的极值满足费马引理(Fermat 引理):若 $ x_0 $ 是局部极值,则 $f(x_0) ' =0$,这个容易通过导数的定义进行证明,如何判断该点是局部极大值还是局部极小值呢?需要用到二阶导,如果二阶导大于0,说明一阶导在单增,一阶导数是从负数变成正数,该函数值是先减小再增大,所以该极值点是局部极小值点,同理,如果二阶导数小于0,说明该点是局部极大值点。如果二阶导数依然为0,但是还想判断出这个点的属性,可以继续求导判断,在实际使用中可以取周围的点算出函数值进行判断。
    
    \item 对于多元函数,同样有费马引理,即如果多元函数 $f$ 在 $P_0$ 可微并且 $f(P_0)$ 是一个局部极值,那么 $\nabla f(P_0) = \vec{0}$。多元函数的二阶导数(如果导数存在)怎么求呢?想到向量值多元函数的求导(微分) ,多元函数求一阶导之后得到的不就是一个向量值多元函数吗?那求二阶导直接使用向量值多元函数的求导,得到称为  Hessian 矩阵的一个矩阵,如下
    $$
    D^{2}f:=\left(
    \begin{array}{ccc}
    f_{11} & \cdots & f_{1n} \\
    \vdots & \ddots & \vdots \\
    f_{n1} & \cdots&f_{nn}
    \end{array}
    \right)_{n\times n}
    $$
    如何判断多元函数的极值点是极大值点还是极小值点呢?就可以用 Hessian 矩阵来判断,如果它正定,那么就是极小值点,如果负定,就是极大值点。矩阵正定和负定的概念在线性代数中介绍过,这里再复习理解一下,也会增加对正定负定的理解。

    \item 对于二元函数,其 Hessian 矩阵为
    $$
    D^2f=\left(\begin{array}{cc}f_{xx}&f_{xy}\\f_{yx}&f_{yy}\end{array}\right)
    $$
    要使得 $D^2f$ 是正定或者负定的,必有 $f_xx > 0 或者 f_xx < 0 $,并且
    $$
    |D^2f|=f_{xx}f_{yy}-f_{xy}^2>0
    $$
    如果 $ |D^2f(P_0)| < 0  $,那么 $P_0$ 一定不是极值点,而是鞍点。

    \item 自由极值与条件极值,自由极值是函数没有约束条件下的极值,条件极值是给自由函数增加一些约束条件(如通过方程限制其变量范围等),将条件极值变成自由极值的一种办法是通过解出限制方程的解,将条件极值问题变成自由极值问题,但是显然这种方法不是通法,因为有些方程是不容易解出显式解的,这时候就可以用到前面学过的隐函数的知识,例如在约束条件
    $$
    g(x,y)=0
    $$
    下求目标函数 $f(x,y)$的极值,如果 $g_y \neq 0 $,则由隐函数定理,方程 $g(x,y)=0$决定了隐函数 $y=y(x)$,于是这个条件极值问题就变成了求函数
    $$
    f(x,y(x))
    $$
    的自由极值问题。Lagrange 乘数法的思路和这个类似,最重要的就是运用了隐函数的思想。

    \item 函数 $f(x,y(x))$ 的可微自由极值点一定满足
    $$
    \frac d{dx}f(x,y(x))=f_x+f_y\frac{dy}{dx}=f_x-f_y\frac{g_x}{g_y}=0
    $$
    这个条件等价于
    $$
    \frac{f_x}{g_x}=\frac{f_y}{g_y}=-\lambda \Leftrightarrow\nabla f= - \lambda\nabla g \Leftrightarrow \nabla f + \lambda\nabla g = 0
    $$
    这等价于说所求极值点是下面构造出的 Lagrange 函数的一个驻点
    $$
    L(x,y;\lambda):=f(x,y)+\lambda g(x,y)
    $$
    求该条件极值时,也就相当于求解方程:
    $$
    \begin{cases}
    \nabla L=\nabla f(x,y)+\lambda\nabla g(x,y)=0 \\
    \partial_\lambda L=g(x,y)=0 
    \end{cases}
    $$
    若同时有多个约束方程,也可以用同样的思路进行处理,例如约束条件如下时
    $$
    G(x)=(g_1(x),\cdots,g_m(x))=\vec{0}
    $$
    构造 Lagrange 函数
    $$
    L(x;\lambda)=f(x)+\vec{\lambda}\cdot G(x)=f(x)+\sum_{i=1}^m\lambda_ig_i(x)
    $$
    目标函数 $f(x)$ 在多个约束条件 $G(x) = \vec{0}$ 下的极值点一定是该Lagrange函数的驻点,求解方程
    $$
    \begin{cases}\nabla L=\nabla f(x,y)+\lambda\nabla g(x,y)=0 \\
    \partial_\lambda L=g(x,y)=0
    \end{cases}
    $$
    即可解得极值点。
\end{enumerate}

\subsubsection{题目示例}
\begin{enumerate}
    \item 求对角线长度为2的长方形的最大面积。
    
    \item 求满足约束条件 $g(x,y,z) = 9x^2 + 4y^2 - z =0$下,目标函数 $f(x,y,z) = 3 x + 2y+z +5$的最小值。

    \item 求 $f(x,y,z)=z + 2y + 3z$在圆柱面 $x^2+y^2=2$和平面 $y+z=1$ 相交得到的椭圆上能取到的最大和最小值。
\end{enumerate}

\subsubsection{习题参考}
\begin{enumerate}
    \item Find the area of the ellipse of intersection of the plane $x+y+z=0$ and the ellipsoid $x^2 + y^2 + 4z^2=1$. 求椭球 $x^2 + y^2 + 4z^2=1$ 与平面 $x+y+z=0$ 相交得到的椭圆的面积。

    \item (教材第121页习题9-8第6题)从斜边之长为 $l$ 的一切直角三角形中,求有最大周长的直角三角形。

    \item (教材第121页习题9-8第7题)要造一个体积等于定数 $k$ 的长方体无盖水池,应如何选择水池的尺寸,方可使它的表面积最小。

    \item (教材第121页习题9-8第10题)求内接于半径为 $a$ 的球且有最大体积的长方体。

    \item (教材第121页习题9-8第11题)抛物面 $z=x^2+y^2$ 被平面 $x+y+z=1$ 截成一椭圆,求这个椭圆上的点到原点的距离的最大值与最小值。
\end{enumerate}

\subsection{多元函数的中值定理和Taylor公式}
\subsubsection{知识概要}
\begin{enumerate}
    \item 先回顾一下一元函数的 Lagrange 中值定理,即:闭区间 $[a, b]$ 上的函数 $f(x)$ 在开区间 $(a, b)$上可导,则存在 $\xi\in(a,b)$,使得
    $$
    f(b) - f(a) = f'(\xi)(b-a)
    $$
    如果还记得Taylor公式,能感受到 Lagrange 中值定理和 Taylor 公式有着非常类似的形式(取最低阶的近似两个公式就一模一样了),Taylor公式是说,$f \in C ^n$,则
    $$
    f(x)=\sum_{k=0}^n\frac{f^{(k)}(x_0)}{k!}(x-x_0)^k+R_n(x)
    $$
    其中 $R_n(x) =o(|x-x_0|^n) $,如果 $f \in C^{n+1}$,则存在 $\xi \in (x_0,x)$,写成 $\xi = x_0 + \theta(x-x_0), \theta \in (0,1)$。
    $$
    R_n(x)=\frac{f^{(n+1)}(\xi)}{(n+1)!}(x-x_0)^{n+1}
    $$

    \item 有了前面的经验,现在就要思考如何把一元的漂亮结果推广到多元呢?使用之前一样的办法,假设 $f:D \to \mathbb{R}$是一个多元函数,给定一个点 $x_0 \in D$和一个单位向量 $\vec{v} \in \mathbb{R}^n $,取一个函数 $\phi(t) = x_0 + t \vec{v}$(这个函数画出了一条线),考虑复合函数 $g(t) = f \circ \phi(t) = f(x_0 + t \vec{v}) $,这样我们就得到了多元函数在某一个方向上的变化,仿若切了一刀,和之前类似把多元函数的复杂变化简单化到一元函数的变化上来了。假设了 $\phi (t)$ 的值域是 $D$ 的子集,并且还需要假设 $D$ 是凸集(即 $\forall x_1, x_2 \in D, x_1+t(x_2-x_1)\in D,\forall t\in[0,1]$,这里可以先不想为何如此),当$f$是可微的,写出$g (t)$的 Lagrange 中值定理(所需要的条件没有写出,但是经过上面的复习应该是明了的):
    $$
    g(t_2) - g(t_1) = g'(\xi) (t_2 - t_1) = \nabla f(x_0 + \xi \vec{v}) \phi'(\xi)(t_2 -t_1) 
    $$
    这等价于多元函数 $f(x)$满足(注意$\phi'(\xi)$是单位向量)
    $$
    f(x_2)-f(x_1)=\nabla f(\xi)\cdot(x_2-x_1), \xi=x_1+\theta(x_2-x_1),\theta\in(0,1)
    $$
    这就是多元函数的中值定理,注意上面的 $x$ 都是向量,所写的点代表点乘。

    \item 类似的办法能不能推广出多元函数的Taylor公式呢?是可以的,得到的形式很类似
    $$
    \begin{aligned}
    f(x)=\sum_{k=0}^n\frac{1}{k!}\nabla ^k f(x_0)\cdot(x-x_0)^k+\frac{1}{(n+1)!}\nabla^{n+1}f(\xi)\cdot(x-x_0)^{n+1}
    \end{aligned}
    $$
    $\nabla$这个符号对于多元函数来说就像一元函数的求导,我们应该不止一次注意到这个类似了,要略微思考上面这个公式的 $\nabla ^k$和 $(x-x_0)^k$表示什么。从几何意义上来说,一元函数的泰勒公式是使用从低阶到高阶的多项式函数(曲线)来对复杂(或者说未知)曲线进行拟合,多元函数的泰勒公式几何意义是一样的,也是使用从低阶到高阶的多项式函数(曲面)来对复杂的曲面进行拟合。
\end{enumerate}

\newpage

\section{多元函数积分}
\subsection{重积分}
\subsubsection{知识概要}
\begin{enumerate}
    \item 一元函数的定积分:一元函数的定积分是在求曲边梯形的面积,通过分割,近似,求和,取极限完成,当求取的极限存在并且与分割方式无关时,称为函数 $f(x)$ 在闭区间 $[a, b]$ 上可积,并把求到的极限称为 $f(x)$ 从 $a$ 到 $b$ 的定积分,记作
    $$
    \int_a^bf(x)dx:=\lim_{\delta\to0}\sum f(\xi_i)\Delta x_i
    $$
    定积分的求取过程是一个求极限的过程,我们可能因为在实际求定积分的过程中大量使用牛顿莱布尼茨公式而忽略定积分的定义。

    \item 将一元的定积分推广到一个闭区间 $ D \subset \mathbb{R}^n$上的多元函数 ,如果是 $n=2$,则几何含义是计算曲面与坐标平面所围成的立体的体积,如果 $D$ 是一个矩形,很容易想到将其分割成若干个小矩形从而计算其体。对于一般区域 $D$,并没有标准的分割方式。此时的解决方案有两种,一种是把函数延拓到更大的矩形区域上,另一种是研究更一般的分割方式,选出可求面积的所谓“可测区域”,当 $D$的边界是分段光滑的曲线组成时,是可测的,两种方案得到的积分值是相同的。

    \item  重积分的定义:像一元积分一样,进行分割、近似、求和、取极限四个步骤,将可测的区域分割成互不相交的可测小区域(分割)
    $$
    D=\Delta\sigma_1\cup\Delta\sigma_2\cup\cdots\cup\Delta\sigma_n
    $$
    选定 $\xi _i \in \sigma_i$并计算每一块的体积近似值(近似)
    $$
    s_i\approx s'_i=f(\xi_i)\cdot|\Delta\sigma_i|, i=1,2,\cdots,n
    $$
    将所有小的体积加起来(求和)
    $$
    S_n=\sum_{i=1}^n s_i'
    $$
    考察 $ \delta=\max\{|(\Delta\sigma_i)|\}\to 0 $ 极限的存在性,并求出极限(取极限)
    $$
    S=\lim_{\delta\to0}S_n=\lim_{\delta\to0}\sum_{i=1}^ns_i'
    $$
    当极限存在并且与分割近似的取法无关时,称为 $f(x)$在区域 $D$上可积,该极限称为函数 $f(x)$ 在区域 $D$ 上的重积分,记为
    $$
    \int_Df(x)d\sigma:=\lim_{\delta\to0}\sum f(\xi_i)|\Delta\sigma_i|
    $$
    如果 $f$ 是可测有界闭区域 $D$ 上的连续函数,则 $f$ 在 $D$ 上可积。

    \item 重积分的性质,类似一元定积分,有如下容易理解的性质,假设函数 $f$ 和 $g$ 在区域 $D$ 上可积,
    \begin{align*}
    &\int_D1dx=m(D) \\
    &\text{If}D=D_1\amalg D_2,\text{then}\int_Df(x)dx=\int_{D_1}f(x)dx+\int_{D_2}f(x)dx \\
    &\forall a,b\in\mathbb{R}, \int_D[af(x)+bg(x)]dx=a\int_Df(x)dx+b\int_Dg(x)dx \\
    &\text{If}f(x)\leq g(x),\forall x\in D,\text{then}\int_Df(x)dx\leq\int_Dg(x)dx \\
    &\left|\int_Df(x)dx\right|\leq\int_D|f(x)|dx \\
    & \int_Df(x)dx=f(\xi)\cdot m(D), \xi \in D
    \end{align*}
    最后一个式子是重积分的积分中值定理,在一元定积分中也有对应。
\end{enumerate}
% \subsubsection{题目示例}
% \subsubsection{习题参考}

\subsection{重积分的计算}
\subsubsection{知识概要}
\begin{enumerate}
    \item 矩形区域的重积分:假设 $D=[a,b]\times[c,d]\subset\mathbb{R}^2$是一个二维矩形区域,并且函数 $f: D \to \mathbb{R} $连续,则
    $$
    \int\int_Df(x,y)\mathrm{d}x\mathrm{d}y=\int_a^b\int_c^df(x,y)dydx=\int_c^d\int_a^bf(x,y)\mathrm{d}x\mathrm{d}y
    $$
    后面两个式子称为累次积分,中间一个式子先积分 $c$ 到 $d$再积分 $a$到 $b$,这可以理解成先固定一个 $x$ 将对应的 $y$的值积分一边(扫一遍),然后再扫一遍 $x$ 对应的区域即积分 $a$到 $b$这样就实现了遍历整个矩形;最后一个式子则是反过来,先扫一遍 $x$然后再扫 $y$。

    \item 有了上面对累次积分次序的理解,有助于对一般情况的理解,即如果区域 $D$ 容易写成
    $$
    D=\{(x,y)\in\mathbb{R}^2|\phi_1(x)\leq y\leq\phi_2(x),a\leq x\leq b\}
    $$
    则先扫一遍 $y$ 再扫一遍 $x$ 即可,也就是计算如下的累次积分
    $$
    \iint_Df=\iint_R\bar{f}=\int_a^bdx\int_c^d\bar{f}(x,y)dy=\int_a^bdx\int_{\phi_1(x)}^{\phi_2(x)}f(x,y)dy
    $$
    如果积分区域 $D$ 容易写成形如
    $$
    D=\{(x,y)\in\mathbb{R}^2|\psi_1(y)\leq x\leq\psi_2(y),c\leq y\leq d\}
    $$
    则先扫一遍 $x$再扫一遍 $y$,也就是计算如下的累次积分
    $$
    \iint_Df(x,y)\mathrm{d}x\mathrm{d}y=\int_c^ddy\int_{\psi_1(y)}^{\psi_2(y)}f(x,y)dx
    $$
    更一般的区域常常可以划分为以上两种分割模式的组合。

    \item 对于三重积分,也就是一个三维区域 $\Omega \subset \mathbb{R}^3$的定积分,其定义与重积分类似,其物理意义可以想象成知道物体密度的函数 $f(x,y,z)$和描述物体边界的方程,然后求其体积。求它的体积显然可以想到两种分割方法,一种是切成很多薄片,然后串起来,另外一种是劈成很多条然后捆起来。方法一对应的 $\Omega$ 形式为
    $$
    \Omega=\{(x,y,z)\in\mathbb{R}^3|(x,y)\in D_z,a\leq z\leq b\}
    $$
    对应化成积分如下,其中先进行一个重积分就是在求一个薄片,再求一个一元积分就是将薄片串起来,
    $$
    \iiint_\Omega f(x,y,z)=\int_a^b\left(\iint_{D_z}f(x,y,z)\mathrm{d}x\mathrm{d}y\right)dz
    $$
    方法二对应的 $\Omega$ 形式为
    $$
    \Omega=\{(x,y,z)\in\mathbb{R}^3|\phi_1(z,y)\leq z\leq\phi_2(x,y),(x,y)\in D\}
    $$
    对应化成的积分如下,其中先进行的一个一元积分就是在求一小条,再求一个重积分就是将若干个小条捆起来,
    $$
    \iiint_\Omega f(x,y,z)=\iint_D\left(\int_{\phi_1(x,y)}^{\phi_2(x,y)}f(x,y,z)dz\right)\mathrm{d}x\mathrm{d}y
    $$
    这两种都要求某个重积分,使用前面已经讨论过的重积分的计算方法即可。
\end{enumerate}

\subsubsection{题目示例}
\begin{enumerate}
    \item 计算累次积分
    \begin{align*}
        (1) \int_3^5dx\int_{-x}^{x^2}(4x+10y)dy \quad & (2) \int_0^1dy\int_0^{y^2}2ye^xdx
    \end{align*}

    \item 计算图形面积
        \begin{enumerate}[(1)]
             \item 平面 $3x + 6y + 4z -12=0$与坐标平面所围城的四面体。
             \item 由旋转抛物面 $z = x^2 + y^2$,圆柱面 $x^2 + y^2 =4$和坐标平面在第一卦限所围成的立体。
         \end{enumerate}

    \item 计算函数 $f(x,y,z)=2xyz$在由抛物柱面 $z=2-x^2/2$和平面 $z=0, y=x$以及 $y=0$ 在第一卦限所围立体 $\Omega$ 上的三重积分。

    \item 设 $\Omega$ 是由平面 $x+2y+z=1$ 在第一卦限所围的区域,计算 $\int _ \Omega x d \sigma$。
\end{enumerate}

\subsubsection{习题参考}
\begin{enumerate}
    \item (教材第157页习题10-2第2题)画出积分区域,并计算下列二重积分
    \begin{enumerate}[(1)]
        \item $\int _D x \sqrt{y} \mathrm{d} \sigma$,其中 $D$ 是由两条抛物线 $y=\sqrt{x}, y =x^2$所围成的闭区域。
        \item $\int _D xy^2 \mathrm{d} \sigma$,其中 $D$ 是由圆周 $x^2 + y^2=4$及 $y$ 轴所围成的右半闭区域。
        \item $\int _D \mathrm{e}^{x+y} \mathrm{d} \sigma$,其中 $D = \{ (x,y)||x|+|y| \le 1 \}$。
        \item $\int _D (x^2 + y^2 -x) \mathrm{d} \sigma$,其中 $D$ 是由直线 $y=2, y=x$及 $y=2x$所围成的闭区域。
    \end{enumerate}

    \item (教材第157页习题10-2第3题)如果二重积分 $\int _D f(x,y) dxdy$的被积函数 $f(x,y)$ 是两个函数 $f_1(x,y), f_2(x,y)$的乘积,即 $f(x,y) = f_1(x) \cdot f_2(y)$,积分区域 $D = \{ (x,y) | a \le x \le b, c \le y \le d \}$,证明这个二重积分等于两个单积分的乘积,即
    \[
        \int _D f_1(x) \cdot f_2 (y) = \left[ \int _a^b f_1(x) \mathrm{d} x \right] \cdot \left[ \int _a ^d f_2(y) \mathrm{d} y \right]
    \]

    \item (教材第157页习题10-2第6题奇数小题)改换下列二次积分的积分次序:
    \begin{enumerate}[(1)]
        \item $\int _0^1 \mathrm{d} y \int _0 ^y f(x,y) \mathrm{d} x$
        \item $\int _0^1 \mathrm{d} y \int _{-\sqrt{1-y^2}}^{\sqrt{1-y^2}} f(x,y) \mathrm{d} x$
        \item $\int _1^\mathrm{e} \mathrm{d} x \int _0 ^{\ln x} f(x,y) \mathrm{d}y$
    \end{enumerate}

    \item (教材第157页习题10-2第7题)设平面薄片所占的闭区域 $D$ 由直线 $x+y=2, y=x$ 和 $x$ 轴所围成的,它的面密度 $\mu (x,y) = x^2 + y^2$,求该薄片的质量。

    \item (教材第158页习题10-2第10题)求曲面 $z=x^2 + 2y^2$及 $z=6-2x^2-y^2$所围成的立体的体积。

    \item (教材第166页习题10-3第1题)化三重积分 $I = \int _\Omega f(x,y,z) dxdydz$为三次积分,其中积分区域 $\Omega$分别是
        \begin{enumerate}[(1)]
            \item 由双曲抛物面 $xy=z$ 及平面 $x+y-1=0, z=0$所围成的闭区域;
            \item 由曲面 $z=x^2+y^2$ 及平面 $z=1$ 所围成的闭区域;
            \item 由曲面 $z= x^2 + 2y^2$及 $z=2-x^2$ 所围成的闭区域;
            \item 由曲面 $cz = xy (c>0), \frac{x^2}{a^2} + \frac{y^2}{b^2}=1, z=0$所围成的在第一卦限内的闭区域。
        \end{enumerate}

    \item (教材第166页习题10-3第2题)设有一物体,占有空间闭区域 $\Omega = \{ (x,y,z) | 0 \le x \le 1, 0 \le y \le 1, 0 \le z \le 1  \}$,在点 $(x,y,z)$处的密度为 $\rho (x,y,z) = x+y+z$,计算该物体的质量。

    \item (教材第167页习题10-3第4题)计算 $\int _\Omega xy^2z^3 dxdydz$,其中 $\Omega$是由曲面 $z=xy$与平面 $y=x,x=1$和 $z=0$ 所围成的闭区域。
\end{enumerate}

\subsection{极坐标下的重积分}
\subsubsection{知识概要}
\begin{enumerate}
    \item 在前面求二重积分时,我们从矩形的积分区域出发进行讨论,因为我们使用的是直角坐标系,如果换成极坐标呢?此时我们如何分割?一个类比的分割方法是依然利用坐标曲线进行分割,这时分割出的一个小块是一个拥有弧状边的梯形,计算这个微元的面积如下
    $$
    \Delta\sigma=\frac12(r+\Delta r)^2\Delta\theta-\frac12r^2\Delta\theta=r\Delta r\Delta\theta+\frac12\Delta r^2\Delta\theta
    $$
    最后一项相对于前面一项是高阶无穷小,可以省略,所以主要部分就是 $\mathrm{d} \sigma = r \mathrm{d}r \mathrm{d} \theta $,所以对于极坐标积分即可写成
    $$
    \int _D f(r, \theta) \mathrm{d} \sigma = \int _D f(r, \theta) r \mathrm{d}r \mathrm{d} \theta
    $$
    可以类比想到,积分区域边界的形状不同时,转化成累次积分的顺序可以不同来简便计算,对于积分区域如 $D=\{(r,\theta)|\phi_1(\theta)\leq r\leq\phi_2(\theta),\alpha\leq\theta\leq\beta\}$,可以先沿着径向积分(径向积分出一条线),然后再积分角度(线动扫出整个积分区域),即
    $$
    \int_Dfd\sigma=\int_\alpha^\beta d\theta\int_{\phi_1(\theta)}^{\phi_2(\theta)}f(r,\theta)rdr
    $$
    对于 $D=\{(r,\theta)|\psi_{1}(r)\leq\theta\leq\psi_{2}(r),a\leq r\leq b\}$ 可以先积分角度(积分出一个弧形),再沿着径向积分,即
    $$
    \iint_Dfd\sigma=\int_a^bdr\int_{\phi_1(r)}^{\psi_2(r)}f(r,\theta)rd\theta
    $$

    \item 对于柱坐标系,其坐标由参数$(r, \theta, z)$标识,有了极坐标下积分微元的表示,很容易写出柱坐标下的体积微元
    $$
    \mathrm{d} V = \mathrm{d} \sigma \cdot \mathrm{d} z = r \mathrm{d} r \mathrm{d} \theta \mathrm{d} z
    $$
    类似上面已经讨论过三重积分,对于不同形状的区域,我们依然可以采用将若干个薄片串起来或者将若干细条捆起来的积分次序进行柱坐标下的三重积分。

    \item 如何求取球坐标系的体积微元呢?可以先尝试画出球坐标系下的体积微元,然后进行计算,球坐标系下的坐标标记为 $(\rho, \theta, \phi)$,其与直角坐标系的转换关系为
    $$
    x=\rho\sin\phi\cos\theta,y=\rho\sin\phi\sin\theta,z=\rho\cos\phi
    $$
    如果画出(或者想象出)球坐标系下的体积微元,其是一个带有球面的长方体,三条边的长度分别为 $\rho \sin \phi \mathrm{d} \theta, \rho \mathrm{d} \phi, \mathrm{d} \rho$,故体积微元为
    $$
    \mathrm{d} V = \rho^2 \sin \phi \mathrm{d} \rho \mathrm{d} \theta \mathrm{d} \phi
    $$
\end{enumerate}

\subsubsection{题目示例}
\begin{enumerate}
    \item 设 $D$ 是第一象限中位于圆 $r=2$的外部,心形线 $r=2(1+ \cos \theta)$的内部的区域,计算
    $\int _D y \mathrm{d} \theta$。

    \item 求抛物面 $z = x^2 + y^2$,圆柱面 $x^2 + y^2 = 2y$ 和 $xy$ 坐标平面所围立体的体积。

    \item 计算高斯积分 $\int _0 ^{\infty} \mathrm{e}^{-x^2} \mathrm{d} x$。

    \item 求第一卦限中由抛物面 $z = 4 - x^2 -y^2$和圆柱面 $x^2+ y^2=2x$所围成三维立体体积。

    \item 计算由球面 $\rho = a$和圆锥面 $\phi = \alpha $所围成的三维立体的体积。

    \item 设 $\Omega$ 是抛物面 $z = x^2 + y^2 $和平面 $z=4$ 所围立体,计算 $\int _D z \mathrm{d} x \mathrm{d} y \mathrm{d} z$。
\end{enumerate}

\subsubsection{习题参考}
\begin{enumerate}
    \item (教材第158页习题10-2第13题)把下列积分化为极坐标形式,并计算积分值
        \begin{enumerate}[(1)]
            \item $\int _0 ^2a \mathrm{d} x \int _0^{\sqrt{2ax-x^2}} (x^2+y^2) \mathrm{d}y $
            \item $\int _0 ^a \mathrm{d} x \int _0^x \sqrt{x^2 + y^2} \mathrm{d}y$
            \item $\int _0^1 \mathrm{d} x \int _{x^2}^x (x^2 + y^2)^{-\frac{1}{2}} \mathrm{d}y  $
            \item $\int _0 ^a \mathrm{d}y \int _0 ^{\sqrt{a^2 -y^2}} (x^2 + y^2) \mathrm{d}x$
        \end{enumerate}

    \item (教材第158页习题10-2第15题)选用适当的坐标系计算下列各题:
        \begin{enumerate}[(1)]
            \item $\int _D \frac{x^2}{y^2} \mathrm{d}\sigma$,其中 $D$ 是由直线 $x=2,y=x$及曲线 $xy=1$ 所围成的闭区域。
            \item $\int _D \sqrt{\frac{1-x^2-y^2}{1+x^2+y^2}} \mathrm{d}\sigma$,其中 $D$ 是由圆周 $x^2+y^2=1$ 及坐标轴所围成的第一象限内的闭区域。
            \item $\int _D (x^2 + y^2) \mathrm{d} \sigma$,其中 $D$ 是由直线 $y=x, y=x+a, y=a, y=3a(a>0)$所围成的闭区域。
            \item $\int _D \sqrt{x^2 +y^2} \mathrm{d} \sigma$,其中 $D$ 是环形闭区域 $\{ (x,y) | a^2 \le x^2 + y^2 \le b^2\}$
        \end{enumerate}

    \item (教材第158页习题10-2第17题) 求由平面 $y=0,y=kx(k>0),z=0$以及球心在原点、半径为 $R$ 的上半球面所围成的在第一卦限内的立体的体积。

    \item (教材第167页习题10-3第6题) 计算 $\int _\Omega xyz dx dy dz$,其中 $\Omega$ 为球面 $x^2 + y^2 +z^2 =1$及三个坐标面所围成的在第一卦限内的闭区域。

    \item (教材第167页习题10-3第8题) 计算 $\int _\Omega z dx dy dz$,其中 $\Omega$ 是由锥面 $z = \frac{h}{R} \sqrt{x^2 + y^2}$与平面 $z=h (R>0, h>0)$所围成的闭区域。

    \item (教材第167页习题10-3第11题) 选用适当的坐标计算下列三重积分
        \begin{enumerate}[(1)]
            \item $\int _\Omega xy \mathrm{d}V$,其中 $\Omega$ 为柱面 $x^2 + y^2=1$及平面 $z=1,z=0,x=0,y=0$所围成的在第一卦限内的闭区域。
            \item $\int _\Omega \sqrt{x^2 + y^2 + z^2} \mathrm{d} V$,其中 $\Omega$是由球面 $x^2+y^2+z^2=z$所围成的闭区域。
            \item $\int _\Omega(x^2 +y^2) \mathrm{d}V$,其中 $\Omega$是由曲面 $4z^2=25(x^2+y^2)$及平面 $z=5$所围成的闭区域。
            \item $\int _\Omega (x^2 + y^2) \mathrm{d} V$,其中闭区域 $\Omega$ 由不等式 $0 < a \le \sqrt{x^2 + y^2 + z^2} \le A, z \ge 0$所确定。
        \end{enumerate}
\end{enumerate}

\subsection{重积分的变量替换}
\subsubsection{知识概要}
\begin{enumerate}
    \item 对于一个一元积分 $\int f(x) \mathrm{d} x$,可以令 $x = g(t)$ 进行变量替换,将原来的积分转换为
    $$
    \int f(x) \mathrm{d} x = \int f(g(t)) \mathrm{d} g(t) = \int f(g(t)) g'(t) \mathrm{d} t
    $$
    这里我们显然需要 $g(t)$ 是可导的,并且 $g'(t) \neq 0$。

    \item 如何把上面的结论推广到多元积分呢?考虑多元积分 $\int f(x) \mathrm{d} \sigma$,这时候 $x$ 是一个向量,应该要找同样维度的向量(同样多的变量)来进行变量替换,也就是说要找一个与 $x$维度相同的向量 $t$,并构造函数 $x = g(t)$,那 $\mathrm{d} \sigma $怎么解决,也就是积分微元如何解决?类比一元的情形,$\mathrm{d} x = g'(t) \mathrm{d} t $,这就是 $x = g(t)$ 这个函数的微分啊,多元积分时也直接将积分微元换成微分是不是就行了?实际上确实如此,但是假设 $x$ 是 $n$ 维向量,则$t$也是一个 $n$维向量,函数 $x = g(t)$是一个 $\mathbb{R}^n \to \mathbb{R}^n $的映射,$\mathrm{D} x = A_{n \times n} \mathrm{D}t$,这个$A$是一个矩阵(雅可比矩阵,Jacobian matrix应该标记为 $J$),在计算积分时,取其行列式即可(二阶行列式的几何意义是平行四边形面积,三阶行列式是平行六面体的体积,这里也取行列式,有什么思考?),这里我们没有严谨地证明,从形式上(或许不严谨地)从一元积分换元类比推导出了多元微积分的换元
    $$
    \int _D f(x) \mathrm{d} x = \int _U f(g(t)) \cdot |J_g(t)| \mathrm{d}t
    $$
    上面依然沿用了一元积分的记号,而且没有加粗,但是从上下文来看不会引起歧义,其中 $J_g(t)$表示多元向量值函数$x = g(t)$的雅可比矩阵(说雅可比矩阵确实值得纪念雅可比,但是也可以想成就是求导), $|J_g(t)|$表示雅可比矩阵对应的行列式。
    $$
    J_g(t) =\frac{\partial(x_1,\cdots,x_n)}{\partial(t_1,\cdots,t_n)}
    $$
    $J_g(t)$的行列式不能为0(回想线性代数中相关的概念,这是对从 $\mathrm{d} t$到 $\mathrm{d} x$的映射做了什么要求)。

    \item 从积分换元的角度看极坐标和球坐标下的积分,顺便练习一下重积分的换元。考虑直角坐标系到极坐标的换元,即 $x = r \cos \theta, y = r \sin \theta$,这里 $J_g$即为
    $$
    \frac{\partial (x, y)}{\partial (r, \theta)} =
    \begin{bmatrix}
    \frac{\partial x}{\partial r}  & \frac{\partial x}{\partial \theta }  \\
    \frac{\partial x}{\partial r}  & \frac{\partial x}{\partial \theta }  
    \end{bmatrix}
    = 
    \begin{bmatrix}
    \cos \theta  & -r \sin  \theta \\
    \sin \theta  & r \cos \theta
    \end{bmatrix}
    $$
    所以将直角坐标系的重积分换为极坐标系的重积分即为
    $$
    \int _D f(x,y) \mathrm{d}x \mathrm{d} y = \int _U f(r\cos \theta, r\sin \theta) \begin{vmatrix}
    \cos \theta  & -r \sin  \theta \\
    \sin \theta  & r \cos \theta
    \end{vmatrix} 
    \mathrm{d} r \mathrm{d} \theta = \int _U f(r\cos \theta, r\sin \theta) r \mathrm{d} r \mathrm{d} \theta
    $$
    和之前得到的结果是一样的,同理将直角坐标系换为球坐标系时,写出其导数(雅可比矩阵)的行列式
    $$
    \left| \frac{\partial(x,y,z)}{\partial(\rho,\theta,\phi)} \right|=\left|\begin{array}{ccc}\cos\theta\sin\phi&-\rho\sin\theta\sin\phi&\rho\cos\theta\cos\phi\\\sin\theta\sin\phi&\rho\cos\theta\sin\phi&\rho\sin\theta\cos\phi\\\cos\phi&0&-\rho\sin\phi\end{array}\right|=-\rho^2\sin\phi,
    $$

    \item 在上面我们提到说二阶行列式的值是平行四边形的面积(哪个平行四边形呢?就是二阶行列式列向量形成的平行四边形),三阶行列式的值是平行六面体的体积,由此可以推广定义 $n$维平行多面体的体积,即 $n$ 维欧氏空间中线性无关的向量 $v_1, v_2, \cdots, v_n \in \mathbb{R}^n$形成的 $n$ 维平行多面体的体积定义为
    $$
    V = |det(v_1, v_2, \cdots, v_n)|
    $$
    这里的竖线是取绝对值,因为一般说到体积是正数。
\end{enumerate}

\subsubsection{题目示例}
\begin{enumerate}
    \item 假设 $D$ 是由直线 $x+y=2$ 与 $x$ 轴,$y$ 轴围成的区域,求 $\int _D \mathrm{e}^{\frac{y-x}{y+x}} \mathrm{d} x \mathrm{d} y$。

    \item 计算直线 $x+y =c, x+y=d, y=ax, y=bx$围出的区域的面积。

    \item 假设 $D$ 是由椭圆 $\frac{x^2}{a^2} + \frac{y^2}{b^2} =1$围成的区域,计算 $\int _D \sqrt{1- \frac{x^2}{a^2} - \frac{y^2}{b^2}} \mathrm{d}x \mathrm{d}y$。

    \item 求由椭球面 $\frac{x^2}{a^2}+\frac{y^2}{b^2}+\frac{z^2}{c^2}=1$ 围成的三维区域的体积。
\end{enumerate}

\subsubsection{习题参考}
\begin{enumerate}
    \item (教材第159页习题10-2第20题) 求由下列曲线所围成的闭区域 $D$ 的面积
    \begin{enumerate}[(1)]
        \item $D$ 是由曲线 $xy=4,xy=8,xy^3=5,xy^3=15$所围成的第一象限部分的闭区域。
        \item $D$ 是由曲线 $y=x^3, y=4x^3, x=y^3, x=4y^3$所围成的第一象限部分的闭区域。 
    \end{enumerate}

    \item (教材第159页习题10-2第21题) 设闭区域 $D$ 是由直线 $x+y=1, x=0, y=0$所围成,求证
    \[
        \int _D \cos \left( \frac{x-y}{x+y} \right) dx dy = \frac{1}{2} \sin 1
    \]
\end{enumerate}

\subsection{重积分的应用}
\subsubsection{知识概要}
\begin{enumerate}
    \item 在学习完一元积分后,我们学会了求曲线的长度,同理,现在我们可以求曲面的面积。假设 $\Sigma$ 是一个二维曲面,我们可能拥有它的显式形式 $z = f(x,y)$,也有可能只知道一般形式,$F(x,y,z)=0$,不过我们丝毫不慌的是知道一般形式可以使用隐函数定理这个利器将其化为显式形式。先考虑显式形式下求曲面的面积。对我们想求面积的区域 $D$ 将其做分割,使用小平行四边形的面积来逼近小曲面的面积(如同我们当初使用直线逼近曲线的长度),小平行四边形的长度为 $f_x(x,y) \mathrm{d}x , f_y(x,y) \mathrm{d}y$,其面积为
    $$
    \mathrm{d}A = |f_x \mathrm{d}x \times f_y \mathrm{d}y| = \left|\begin{array}{ccc}\mathbf{i}&\mathbf{j}&\mathbf{k}\\1&0&f_x\\0&1&f_y\end{array}\right|\mathrm{d}x\mathrm{d}y=|(-f_x,-f_y,1)|\mathrm{d}x\mathrm{d}y=\sqrt{1+f_x^2+f_y^2}\mathrm{d}x\mathrm{d}y.
    $$

    \item 给出曲面一般形式 $F(x,y,z)=0$时,我们使用隐函数定理表达 $f_x$和 $f_y$,隐函数定理的结论已经模糊,重新简单推导即可,将 $z$ 看成是 $x,y$ 的函数,对一般形式的 $x$ 和 $y$ 求偏导
    $$
    \frac{\partial F}{\partial x} + \frac{\partial F}{\partial z}\frac{\partial z}{\partial x} = 0 , \quad \frac{\partial F}{\partial y} + \frac{\partial F}{\partial z}\frac{\partial z}{\partial y} = 0 
    $$
    即可得到
    $$
    \frac{\partial z}{\partial x} = - \frac{F_x}{F_z}, \quad \frac{\partial z}{\partial y} = - \frac{F_y}{F_z}
    $$
    从而得到此时的面积微元表达式
    $$
    dA=\sqrt{1+f_x^2+f_y^2}\mathrm{d}x\mathrm{d}y=\frac{|\nabla F|}{|F_z|}\mathrm{d}x\mathrm{d}y, \quad A=\int_D\sqrt{1+f_x^2+f_y^2}\mathrm{d}x\mathrm{d}y=\int_D\frac{|\nabla F|}{|F_z|}\mathrm{d}x\mathrm{d}y
    $$
    这和我们当初学完一元积分时计算曲线的长度是类似的,曾经给定一段区间,知道这段区间上的曲线方程,我们可以计算出曲线的长度,现在给定一块区域,知道这块区域上曲面的方程我们就可以计算曲面的面积。

    \item 曲面面积的几何解释,求曲面面积时如果知道的是曲面的一般形式,得出的公式中有 $\frac{|\nabla F|}{|F_z|}$这一项,如果从公式中去掉这一项得到的就是划定区域的面积,说明这一项是微元中小平行四边形面积 $S'$和下方对应矩形面积 $S$ 的比值,也就是说 $\frac{|f_x \mathrm{d} x \times f_y \mathrm{d}y| }{\mathrm{d} x \mathrm{d}y} = \frac{|\nabla F|}{|F_z|}$, 我们知道 $\nabla F$是曲面的法向量,$F_z$是 $\nabla F$的 $z$轴分量,他们的大小关系可以使用他们之间的夹角来描述
    $$
    \nabla F\cdot(0,0,1)=F_z=|\nabla F|\cos\theta
    $$
    $F_z$ 也可以认为是 $xy$ 平面的法向量,平面之间的二面角是等于法向量之间的夹角的,所以小平行四边形和底面矩形的夹角为 $\theta$, 小平行四边形与矩形的面积关系为 $S = S' \cos \theta$
    故
    $$
    dA=\frac{1}{|\cos\theta|}dxdy=\frac{|\nabla F|}{|F_z|}dxdy
    $$
    在这里稍微故意强调说提到面积和体积时数字的值为正值,实际上后面会学到曲面积分,为方便理解提出有向面积的概念,也就是叉乘得到的向量作为有向面积,这个向量的大小(大于0)也就是我们这里反复说的面积。

    \item 引出三重积分时以求物体的质量为例子,有了重积分这个工具,我们可以非常容易计算物体的质心以及转动惯量之类,例如已知密度函数 $u(x,y,z)$,则其质量的微分非常容易表达为
    $$
    u(x,y,z) \mathrm{d}\sigma
    $$
    显然 $x_i$ 个轴的质心计算公式为
    $$
    \bar{x}_i = \frac{\int _ D x_i u(x,y,z) \mathrm{d}\sigma}{\int _D u(x,y,z) \mathrm{d}\sigma}
    $$
    $x$轴的转动惯量为
    $$
    I_x = \int _D (y^2 + z^2) u(x,y,z) \mathrm{d}\sigma
    $$
    同理可以写出其他轴的转动惯量计算公式。如果对质心的定义或者转动惯量的定义不熟悉了,请复习物理课中对应的内容。
\end{enumerate}

\subsubsection{题目示例}
\begin{enumerate}
    \item 求半径为 $a$ 的二维球面的面积。

    \item 计算旋转抛物面 $z = x^2 + y^2$位于平面 $z=9$下方部分的面积。

    \item 计算 $x^2 + y^2 + z^2 =a^2$ 被圆柱面 $x^2 + y^2 = ax$ 所围部分的面积。

    \item 假设一个均匀平面薄片由两个圆 $r=2\sin \theta$和 $r = 4\sin \theta$围成,求质心。

    \item 求一个具有常密度的半球体的质心位置。

    \item 求半径为 $a$ 的均匀半圆薄片对于其直径边的转动惯量。
\end{enumerate}

\subsubsection{习题参考}
\begin{enumerate}
    \item (教材第177页习题10-4第2题) 求锥面 $z=\sqrt{x^2 + y^2}$被柱面 $z^2 = 2x$所割下部分的曲面面积。
    
    \item (教材第178页习题10-4第5题)  设薄片所占的闭区域 $D$ 由抛物线 $y=x^2$ 及直线 $y=x$ 所围成,它在点 $(x,y)$ 处的面密度为 $\mu (x,y) =x^2y$,求该薄片的质心。
    
    \item (教材第178页习题10-4第10题) 已知均匀矩形板(面密度为为常量 $\mu$)的长和宽分别为 $b$和 $h$,计算此矩形板对于通过其形心且分别与一边平行的两轴的转动惯量。
\end{enumerate}

\newpage

\section{曲线曲面积分}
\subsection{引入与曲线积分}
\subsubsection{知识概要}
\begin{enumerate}
    \item 在这一节课中,老师提纲挈领讲述了曲线积分和曲面积分的全貌。曲线积分是将定积分从直线上推广到曲线上,曲面积分是平面上的重积分推广到曲面上,不难发现之前做一元定积分时都是 $x$ 从某个值积分到某个值,这是一段直线定积分,因为是一维的,重积分也是在二维平面上进行。实际运用中完全有曲线曲面的情况,在物理的电磁学中应该尤其有这样的感受。现在对曲线和曲面积分进行一些一般性的研究,会发现这是非常有趣的内容。

    \item 向量场:场的概念非常基本并且非常重要,可以简单分为标量场和向量场,空间中的温度分布就是一个标量场,温度这个物理量只有大小没有方向,这个场的方程是一个三元函数 $f(x,y,z)$;但是空间中的速度场就是一个向量场,因为速度是矢量,向量场的方程是一个多元向量值函数,$F(x,y,z) = (P(x,y,z,Q(x,y,z), R(x,y,z))$。为方便后文叙述,我们约定对于三维的向量场其分量写成 $(P,Q,R)$,对于二维的向量场,其分量写成 $F(x,y)= ( P(x,y), Q(x,y) )$

    \item 单位向量的方向余弦,给定 $n$ 维欧氏空间中的一个向量 $T = (x_1, x_2, \cdots, x_n) \in \mathbb{R}^n$,它与第 $i$ 个坐标轴正向方向的夹角为 $\theta \in [0, \pi]$,且
    $$
    \cos \theta _i = \frac{x_i}{|T _i|}, \quad i=1,2,\cdots, n
    $$
    对于 $n$ 维空间中的单位向量 $t$ ,则显然它与各个坐标轴正方向之间的夹角余弦有如下关系
    $$
    t = (\cos \theta _1, \cos \theta _2, \cdots, \cos \theta _n)
    $$
    显然空间中某个向量与各个坐标轴之间的夹角的余弦值平方和为1,即
    \[
        \cos^2\theta_1+\cos^2\theta_2+\cdots+\cos^2\theta_n= \Sigma_{i=1}^n \cos ^2 \theta _i = 1 
    \]
    
    一般将二维中的单位向量记作 $(\cos \alpha, \cos \beta)$,将三维空间中的单位向量记作 $(\cos \alpha, \cos \beta, \cos \gamma)$

    \item 平面曲线的切向量,一条平面曲线可以表示为一个二维的向量值函数,$\mathbf{r}: t \to (x(t), y(t))$,有向曲线元也就是 $\mathbf{r}$的微分
    \[
        \mathrm{d} \mathbf{r} = (\mathrm{d}x, \mathrm{d}y) = \mathbf{r}' \mathrm{d}t = \mathbf{t} \mathrm{d}s
    \]
    其中 $\mathrm{d}s$是指弧长微元,即
    \[
    \mathrm{d}s = |\mathrm{d} \mathbf{r}| = \sqrt{dx^2+dy^2}=\sqrt{(x^{\prime})^2+(y^{\prime})^2}dt
    \]
    粗写的 $\mathbf{t}$ 指的是单位切向量,从上面的式子可以看出
    \[
        \mathbf t=\frac{d\mathbf r}{|d\mathbf r|}=\left(\frac{dx}{ds},\frac{dy}{ds}\right)=(\cos\alpha,\cos\beta)
    \]
    这里写成 $(\cos\alpha,\cos\beta)$ 是提醒说二维单位向量的表示,并没有其他特殊的含义。

    \item 有了平面曲线切向量的概念,我们很容易引出空间曲线的切向量的概念,即对应的有向曲线微元
    \[
        d\mathbf{r}=(dx,dy,dz)=(x',y',z')dt=\mathbf{r}'dt=\mathbf{t}ds
    \]
    弧长微元
    \[
        ds=|d\mathbf{r}|=\sqrt{dx^2+dy^2+dz^2}=\sqrt{(x')^2+(y')^2+(z')^2}dt
    \]
    单位切向量
    \[
    \mathbf{t}=\frac{d\mathbf{r}}{|d\mathbf{r}|}=\begin{pmatrix}\frac{dx}{ds},\frac{dy}{ds},\frac{dz}{ds}\end{pmatrix}=(\cos\alpha,\cos\beta,\cos\gamma)
    \]
    同样的,这里 $(\cos\alpha,\cos\beta,\cos\gamma)$也只是记号。

    \item 曲面法向量,前面提醒过,曲面必须要有两个自由变量,所以空间中的曲面的参数方程可以写为 $\mathbf{r}: (u,v) \to (x(u,v), y(u,v), z(u,v))$,一个面积微元也就是我们前面求曲面面积时的积分微元为
    \[
        dS=|\mathbf{r}_u\times \mathbf{r}_v| \mathrm{d}u \mathrm{d}v = \frac{\mathrm{d}u \mathrm{d}v}{\cos \theta}
    \]
    其中 $\theta$ 是曲面微元的法向量与 $uv$ 平面法向量之间的夹角,也就是曲面微元与平面 $uv$ 之间的夹角。
    如果是 $\mathbf{r}: (x,y) \to (x(x,y), y(x,y), z(x,y))$,则曲面面积微元的法向量为 $N=\mathbf{r}_x \times \mathbf{r}_y$,将其单位化后记作
    \[
        \mathbf{n} = \frac{\mathbf{N}}{|\mathbf{N}|}= (\cos \alpha, \cos \beta, \cos \gamma) = \frac{(F_x, F_y, F_z)}{|\nabla F|}
    \]
    记有向曲面元为
    \[
        d \mathbf{S} = \mathbf{n} \mathrm{d} S = (\cos \alpha, \cos \beta, \cos \gamma) \mathrm{d} S =(\pm \mathrm{d}\sigma_{yz}, \pm \mathrm{d}\sigma_{zx}, \pm \mathrm{d}\sigma_{xy})
    \]
    曲面微元 $\mathrm{d}S$与在三个面上的投影的面积关系为(正负号因 $\cos \theta $产生)
    \[
        \mathrm{d}S = \left|\frac{\mathrm{d} \sigma _{yz}}{\cos \alpha} \right| 
        = \left| \frac{\mathrm{d} \sigma_{zx}}{\cos \beta} \right|
        = \left| \frac{\mathrm{d} \sigma _{xy}}{\cos \gamma} \right|
    \]
    上面这几点内容是复习,要熟练掌握,因为是学习曲线积分和曲面积分的基础。

    \item 第一类曲线积分,被积函数为一个标量函数,积分微元也是一个标量,沿着曲线进行积分即
    \[
        \int _L f \mathrm{d} s = \lim_{\lambda\to0}\sum_if(\xi_i)\Delta s_i = \int _a^b f(\mathbf{r}(t)) \sqrt{(x'_1)^2+(x'_2)^2 + \cdots + (x'_n)^2} \mathrm{d}t
    \]
    其中一般从较小的端点 $a$ 积分到较大的端点 $b$,对于二维和三维具体的公式,进行代入即可。第一类曲线积分的被积函数是一个标量函数,产生这种积分的出发点容易想到是计算密度不均匀的曲线的质量。

    \item 第二类曲线积分的被积函数是一个向量,积分微元也是一个向量,沿着曲线进行积分即
    \[
        \int _L \mathbf{F} \cdot \mathrm{d} \mathbf{r} = \int _L (f_1 \mathrm{d}x_1, f_2\mathrm{d}x_2, \cdots, f_n \mathrm{d}x_n) = \int _a^b (f_1 x'_1, f_2 x'_2, \cdots, f_n x'_n)\mathrm{d}t
    \]
    这里 $a$ 和 $b$ 的值是根据积分起点和终点选定的。第二类曲线积分的简单物理对应是沿着曲线计算变力做功,它与第一类曲线积分的明显不同是被积函数的不同,即被积函数是一个向量。第二类曲线积分有时也被称为对坐标的曲线积分。两类曲线积分的总的形式就如此简单,在实际题目中看到会懵的原因主要是分辨不清楚是哪一种积分并且形式没有总的形式这样显然,可能会辨认不出来,不过理解了最基础的,具体题目多练习几道就没问题,需要有一个熟悉过程。

    \item 曲线积分常见的性质有,对于第一类曲线积分来说
    \[
        \int _L 1 \mathrm{d}s = S
    \]
    被积函数为1的时候得到的就是曲线的长度,如果曲线被分为多段,则可以逐段相加
    \[
        \int _L f \mathrm{d} s = \int _{L_1} f \mathrm{d}s + \int _{L_2} f \mathrm{d}s, \quad L = L_1 + L_2
    \]
    这个对于两类曲线积分来说都成立。第二类曲线积分是有方向的,所以
    \[
        \int_{-L}\mathbf{F}\cdot  \mathrm{d} \mathbf{r}=-\int_L\mathbf{F}\cdot 
     \mathrm{d} \mathbf{r}
    \]

    \item 两类曲线积分的关系,对于第二类曲线积分,由于 $\mathrm{d} \mathbf{r} = \mathbf{t} \mathrm{d}s$,所以有
    \[
        \int _L \mathbf{F} \cdot \mathrm{d} \mathbf{r} = \int _L \mathbf{F} \cdot \mathbf{t} \mathrm{d}s = \int _L (\mathbf{F} \cdot \mathbf{t})\mathrm{d}s = \int _L (f_1 \cos \theta _1+ f_2 \cos \theta _2 + \cdots + f_n \cos \theta _n)\mathrm{d}s
    \]
    这是两类曲线积分的联系,即计算第二类曲线积分时在已知微元与各个轴的余弦值时,很容易将第二类曲线积分转化为第一类曲线积分。
\end{enumerate}

\subsubsection{题目示例}
\begin{enumerate}
    \item 计算 $\int _L xy^4 \mathrm{d}s$,其中 $L$ 是圆 $x^2+y^2=16$的右半边。

    \item 计算 $\int _L xyz \mathrm{d} s $, 其中 $L$ 是螺旋线 $\mathbf{r}(t) =(\cos t , \sin t, 3t), 1 \le t \le 4 \pi$。

    \item 计算 $\int _L xy \mathrm{d}x$,其中 $L$ 是抛物线 $y^2=x$,定向从 $A(1,-1)$到 $B(1,1)$。

    \item 计算 $\int _L \mathbf{F} \cdot \mathrm{d} \mathbf{r}$,其中 $\mathbf{F} = (8x^2yz, 5z, -4xy)$, $L$是曲线 $\mathbf{r}=(t,t^2,t^3)$,定向从 $t=0$ 到 $t=1$。

    \item 计算 $\int _L x^3 \mathrm{d} x + 3zy \mathrm{d}y - x^2y \mathrm{d}z$,其中 $L$ 是从点 $A(3,2,1)$ 到点 $B(0,0,0)$ 的直线段。

    \item 计算 $\int _L y^2 \mathrm{d}x$,其中 $L$为(1) 从 $A(a,0)$到 $B(-a,0)$的单位圆上半圆; (2)从 $A(a,0)$到 $B(-a,0)$的直线段。

    \item 计算 $\int _L 2xy \mathrm{d}x + x^2 \mathrm{d}y$,其中 $L$ 为 (1)抛物线 $y=x^2$从 $O(0,0)$到 $B(1,1)$的有向弧;(2)抛物线 $x=y^2$上从 $O(0,0)$ 到 $B(1,1)$的有向弧; (3)从 $O(0,0)$到 $A(1,0$再到 $B(1,1)$的有向折线。
\end{enumerate}
从上面的例子中我们可以看出,曲线积分的值可能依赖于路径的选择。

\subsubsection{习题参考}
\begin{enumerate}
\item (教材第193页习题11-1第3题偶数题) 计算下列对弧长的曲线积分
    \begin{enumerate}[(1)]
        \item $\int _L (x+y) \mathrm{d} s$,其中 $L$ 为连接 $(1,0)$ 及 $(0,1)$两点的线段。
        \item $\oint _L \mathrm{e}^{\sqrt{x^2 + y^2}} \mathrm{d}s$,其中 $L$ 为圆周 $x^2+y^2 =a^2$,直线 $y=x$以及 $x$轴在第一象限内所围成的扇形的整个边界。
        \item $\int _\Gamma x^2 yz \mathrm{d}s$,其中 $\Gamma$为折线 ABCD,这里 A,B,C,D依次为点 $(0,0,0),(0,0,2),(1,0,2),(1,3,2)$。
        \item $\int _L (x^2 + y^2 ) \mathrm{d}s$,其中 $L$ 为曲线 $x=a(\cos t + t\sin t), y = a(\sin t - t \cos t) (0 \le t \le 2 \pi)$。
    \end{enumerate}

\item (教材第193页习题11-1第5题) 设螺旋形弹簧一圈的方程为 $x=a \cos t, y= a\sin t, z=kt(0 \le t \le 2\pi)$,它的线密度 $\rho(x,y,z) = x^2+y^2+z^2$,求:
    \begin{enumerate}[(1)]
        \item 它关于 $z$ 轴的转动惯量 $I_Z$。
        \item 它的质心。
    \end{enumerate}

\item (教材第203页习题11-2第3题奇数题) 计算下列对坐标的曲线积分
    \begin{enumerate}[(1)]
        \item $\int _L (x^2 - y^2) \mathrm{d}x$,其中 $L$ 是抛物线 $y=x^2$ 上从点 $(0,0)$到点 $(2,4)$的一段弧。
        
        \item $\int _L y \mathrm{d} x + x \mathrm{d}y$,其中 $L$ 为圆周 $x = R \cos t, y = R \sin t$上对应 $t$ 从 0 到 $\frac{\pi}{2}$的一段弧。
        
        \item $\int _\Gamma x^2 \mathrm{d}x + z \mathrm{d}y -y \mathrm{d}z $,其中 $\Gamma$为曲线 $x = k \theta, y= a \cos \theta, z = a \sin \theta$上对应 $\theta$从 0 到 $\pi$的一段弧。

        \item $\int _\Gamma \mathrm{d} x - \mathrm{d}y + y \mathrm{d}z $,其中 $\Gamma$为有向闭折线ABCA,这里的A,B,C依次为 $(1,0,0),(0,1,0),(0,0,1)$。
    \end{enumerate}

\item (教材第204页习题11-2第5题) 一力场由沿横轴正方向的恒力 $\mathbf{F}$ 所构成,试求当一质量为 $m$ 的质点沿圆周 $x^2 + y^2 =R^2$ 按逆时针方向移过位于第一象限的那一段弧时场力所做的功。

\item (教材第204页习题11-2第7题) 把对坐标的曲线积分 $\int _L P(x,y) dx + Q(x,y) dy$化成对弧长的曲线积分,其中$L$为
    \begin{enumerate}[(1)]
        \item 在 $xOy$面内沿直线从点 $(0,0)$ 到点 $(1,1)$。
        \item 沿抛物线 $y=x^2$ 从点 $(0,0)$ 到点 $(1,1)$。
        \item 沿上半圆周 $x^2 + y^2 = 2x$从点 $(0,0)$ 到点 $(1,1)$。
    \end{enumerate}

\end{enumerate}
\subsection{格林公式}
\subsubsection{知识概要}
\begin{enumerate}
    \item  在上面的习题当中,我们感受到了第二类曲线积分有时与积分路径有关,有时与积分路径无关,关键看被积函数,那被积函数满足什么条件时积分与路径无关呢?一元微积分时学到的微积分基本定理(牛顿-莱布尼茨公式)能推广到一般的曲线积分吗?
    \[
        \int_a^bf(x)dx=F(b)-F(a)
    \]
    不定积分能推广吗?
    \[
        F(x)=\int_a^xf(t)dt+C
    \]

    \item 我们知道,不定积分找原函数时可以如下理解,微分 $\mathrm{d} F(x) = f(x) \mathrm{d}x$,相当于$F'(x) = f(x)$
    \[
        \int f(x) \mathrm{d} x = \int \mathrm{d} F(x) = F(x)
    \]
    尝试将这个推广到多元积分呢?多元积分时 $ \mathrm{d} F(\mathbf{x}) = \nabla F \cdot  \mathrm{d} \mathbf{x}$,由此应该有
    \[
        \int \nabla F \cdot  \mathrm{d} \mathbf{x} = \int \mathrm{d} F(\mathbf{x}) = F(\mathbf{x})
    \]
    这很接近曲线积分的积分定理,即如果$L$ 是一条分段光滑曲线 $\mathbf{r}(t): [a,b] \to \mathbb{R}^n $,定向从 $\mathbf{a} = \mathbf{r}(a)$到 $\mathbf{b} = \mathbf{r}(b)$,如果 $f$ 在包含 $L$ 的一个开集内是 $C^{1}$ 的,则
    \[
        \int_L \mathrm{d} f=\int_L\nabla f\cdot \mathrm{d}\mathbf{r}=f(\mathbf{b})-f(\mathbf{a})
    \]
    这个定理的证明可以像之前一样,构造一个一元的映射,然后使用一元微积分中的牛顿莱布尼茨公式进行严格一些的证明。

    \item 从上面讨论到的曲线积分的积分定理可以看出,对于被积函数 $\mathbf{F}(\mathbf{x})$如果它是某个函数的 “导数” 即存在一个 $f$ 满足 $\nabla f(\mathbf{x}) = \mathbf{F}(\mathbf{x})$ 时,它的曲线积分是与路径无关的,只要给定起点 $\mathbf{a}$ 和终点 $\mathbf{b}$,积分值就给定了 $f(\mathbf{b}) - f(\mathbf{a})$ (这和重力做功与路径无好像很类似),类比称 $f(\mathbf{x})$ 为场 $\mathbf{F}(\mathbf{x})$ 的势函数
    \[
        f(\mathbf{x}) = \int _{\mathbf{a}}^\mathbf{x} \mathbf{F} \cdot \mathrm{d} \mathbf{r} + C
    \]
    其中 $\mathbf{a}$ 是一个定点,$C \in \mathbb{R}^1$是一个常数。

    \item 曲线积分与路径无关的条件,假设 $\mathbf{F}$ 是开区域 $D$ 上的连续向量场,以下条件等价:
    \begin{itemize}
    	\item 存在势函数 $f$, 使得 $\mathbf{F} = \nabla f$
            \item 积分 $\int _L \mathbf{F} \cdot \mathrm{d} \mathbf{r}$在 $D$ 中与路径无关
            \item 对任意 $D$ 中的闭光滑曲线 $L$, 都有 $\oint _L \mathbf{F} \cdot \mathrm{d} \mathbf{r} = 0 $
    \end{itemize}
    积分符号上画个圈代表闭曲线上的积分,同时我们称一个向量场为保守(conservative)场,当它存在势函数 $f$ 使得 $\mathbf{F} = \nabla f$,常见的引力场和电场我们都已经学过重力势能和电势能。

    \item 给我一个场的函数,我如何去判断这个场是不是保守场呢?回顾一下它的关键条件即存在一个势函数 $f$ 使得 $\nabla f = \mathbf{F} = (P,Q) = (\frac{\partial f}{\partial x}, \frac{\partial f}{\partial y})$, 给我一个向量场我可以通过解后面这个等式构成的方程将势函数解出来,但是每次要解微分方程,这是不是太难了,有什么其他更好的办法吗?在前面学习偏导数时我们知道如果混合偏导数 $\frac{\partial^2f}{\partial x\partial y}$ 和 $\frac{\partial^2f}{\partial y\partial x}$都存在并且在某一点 $(x_0,y_0)$连续时,有
    \[
        \frac{\partial^2f}{\partial x\partial y}(x_0,y_0)=\frac{\partial^2f}{\partial y\partial x}(x_0,y_0)
    \]
    如果一个场的势函数存在,我们当然期望它的性质足够好(例如期望并且相信重力势能函数,电场势能函数)性质足够好,所以大胆相信势函数存在的话它的混合偏导数是处处连续的,也就是说有如下关系成立
    \[
        \frac{\partial^2f}{\partial x\partial y} = \frac{\partial^2f}{\partial y\partial x}
    \]
    在已知势场函数的情况下,验证这个等式比解微分方程要容易太多,即验证
    \[
        \frac{\partial P}{\partial y}=\frac{\partial Q}{\partial x}
    \]
    由此,我们便容易理解如下定理,假设 $\mathbf{F} = (P_1, P_2, \cdots, P_n)$是单连通区域 $D \subset \mathbb{R}^n$上的一个 $C^1$ 向量场,则 $\mathbf{F}$ 是保守场(i.e. $\mathbf{F} = \nabla f$)当且仅当
    \[
        \frac{\partial P_i}{\partial x_j}=\frac{\partial P_j}{\partial x_i},\quad\forall1\leq i,j\leq n
    \]
    这个定理的必要性证明只要我们将前面的思路倒过来叙述一遍即可,充分性可以使用 Green 公式 和 Stokes 公式来证明。其中单连通区域指的是没有 “洞” 的连通集合(我们提到过的引力场,电场是在哪个集合上的,这个集合上可不可能有 “洞”? ),严格的叙述即为如果 $D$ 中任意闭曲线能够在 $D$ 中连续收缩到一个点。

    \item 格林公式,假设区域 $D \subset \mathbb{R}^2$ 的边界时一条分段光滑曲线 $L$, 记场 $\mathbf{F} = (P, Q) \in C^1(D)$, 则
    \[
        \int_L\mathbf{F}\cdot d\mathbf{r}=\int_LPdx+Qdy=\iint_D\left(\frac{\partial Q}{\partial x}-\frac{\partial P}{\partial y}\right)dxdy
    \]
    这里积分的方向称为“正向”,在之后没有特别说明也都是,所谓正向即观察者在 $L$ 上沿着该方向上走时,始终保持着 $D$ 在观察者的左侧,如果画了一个单位圆为区域 $D$,则正向就是逆时针沿着边缘走。格林公式将沿着某个边缘进行的积分(计算)转化成了计算这个区域内的东西。容易理解以下推论,在某些时候计算区域的面积时带来方便
    \[
        \iint_D1dxdy=\oint_Lxdy=-\oint_Lydx=\frac{1}{2}\oint_Lxdy-ydx.
    \]

    \item 格林公式在另外一种积分中的应用(格林公式的向量形式),假设 $L$ 是二维区域 $D$ 的边界曲线,记 $\mathbf{n}$ 为 $L$ 上指向区域 $D$ 外侧的单位法向量,考虑积分 $\oint _L \mathbf{F} \cdot \mathbf{n} \mathrm{d}s$,解决这个积分可以利用法向量和单位切向量 $\mathbf{t} = (\cos \alpha, \cos \beta)$ 的关系(可以想象单位圆的切向量和法向量来帮助理解)
    \[
        \mathbf{n} = \left(\cos(\alpha-\frac{\pi}{2}),\cos(\beta+\frac{\pi}{2})\right)=(\sin\alpha,-\sin\beta)=(\cos\beta,-\cos\alpha)
    \]
    由此
    \[
        \oint _L \mathbf{F} \cdot \mathbf{n} \mathrm{d}s = \oint _L (P, Q) \cdot (\cos\beta,-\cos\alpha) \mathrm{d}s = \oint _L P \mathrm{d}y - Q \mathrm{d}x = \int _D \left(\frac{\partial P}{\partial x}+\frac{\partial Q}{\partial y}\right) \mathrm{d}x \mathrm{d}y
    \]
    最后一个等式是由格林公式得到的,对于二维向量场 $\mathbf{F} = (P,Q) \in C^1$ 定义其散度
    \[
        \mathbf{divF} = \nabla \cdot \mathbf{F} = \frac{\partial P}{\partial x}+\frac{\partial Q}{\partial y}
    \]
    格林公式的向量形式是指
    \[
    \oint_L\mathbf{F}\cdot\mathbf{n}ds=\iint_D\mathbf{div}\mathbf{F}d\sigma
    \]
\end{enumerate}

\subsubsection{题目示例}
\begin{enumerate}
    \item 计算 $\oint _L x^2y \mathrm{d}x -xy^2 \mathrm{d}y$,其中 $L$ 是圆 $x^2 + y^2 = a^2$取逆时针定向。

    \item 计算 $\int \mathrm{e} ^{-y^2} \mathrm{d}x \mathrm{d}y$,其中 $D$ 是由点 $O(0,0), A(1,1), B(0,1)$确定的三角形区域。

    \item 计算椭圆 $x = a \cos \theta , y = b \sin \theta $的面积。

    \item 假定 $L$ 是一条不过原点的简单闭曲线(无自交点), 取逆时针为定向,计算 $\oint_L\frac{xdy-ydx}{x^2+y^2}$

    \item 证明 $\mathbf{F}(x,y) =\left(-\frac y{x^2+y^2},\frac x{x^2+y^2}\right) $是右半平面 $(x>0)$ 上的保守场,并找到势函数。
\end{enumerate}

\subsubsection{习题参考}
\begin{enumerate}
\item (教材第217页习题11-3第2题) 利用曲线积分,求下列曲线所围成的图形的面积:
    \begin{enumerate}[(1)]
        \item 星形线 $x = a \cos^3 t, y=a \sin^3 t $。
        \item 椭圆 $9x^2 + 16y^2 =144$。
        \item 圆 $x^2 + y^2 =2ax$
    \end{enumerate}

\item (教材第217页习题11-3第6题) 证明下列曲线积分在整个 $xOy$ 面内与路径无关,并计算积分值
    \begin{enumerate}[(1)]
        \item $\int _{(1,1)}^{(2,3)} (x+y) dx + (x-y)dy$。
        \item $\int _{(1,2)}^{(3,4)} (6xy^2 - y^3) dx + (6x^2y -3xy^2) dy$。
        \item $\int _{(1,0)}^{(2,1)} (2xy - y^4 +3)dx + (x^2 - 4xy^3)dy $。
    \end{enumerate}

\item (教材第217页习题11-3第7题奇数题) 利用格林公式,计算下列曲线积分:
    \begin{enumerate}[(1)]
        \item $\oint _L (2x-y+4) dx + (5y + 3x -6) dy$,其中 $L$ 是三顶点分别为 $(0,0),(3,0),(3,2)$的三角形正向边界。
        \item $\int _L (2xy^3 - y^2 \cos x)dx + (1-2y\sin x + 3x^2y^2)dy$,其中 $L$ 为在抛物线 $2x = \pi y^2$上由点 $(0,0)$ 到 $(\frac{\pi}{2}, 1)$的一段弧。
    \end{enumerate}

\item (教材第217页习题11-3第8题偶数题) 验证下列 $P(x,y)dx + Q(x,y)dy$在整个 $xOy$ 平面内是某一函数 $u(x,y)$ 的全微分,并求这样的一个 $u(x,y)$:
    \begin{enumerate}[(1)]
        \item $2xydx + x^2 dy$
        \item $(3x^2y + 8xy^2)dx + (x^3 + 8x^2y + 12y \mathrm{e}^{y})dy$
    \end{enumerate}

\end{enumerate}

\subsection{曲面积分}
\subsubsection{知识概要}
\begin{enumerate}
    \item 第一类曲面积分,被积函数为一个标量函数,积分微元也是标量,将曲面进行积分即为
    \[
        \int _S f \mathrm{d} S = \lim_{\lambda\to 0}\sum_if(\xi_i,\eta_i)\Delta S_i
    \]
    如果被积函数是1得到的就是曲面的面积,给定曲面不同的曲面形式时 $\mathrm{d} S$表达可以不同,当给定参数形式时 $\mathbf{r}(u,v)=(x(u,v),y(u,v),z(u,v)),(u,v)\in U$
    \[
        \int _S f dS=\int_Uf(\mathbf{r}(u,v))|\mathbf{r}_u\times\mathbf{r}_v|dudv
    \]
    当给显式形式 $z = z(x,y), (x,y) \in D$ 时
    \[
        \int_SfdS=\int_Df(x,y,z(x,y))\sqrt{z_x^2+z_y^2+1}dxdy
    \]
    当给定一般形式时
    \[
        \int _SfdS=\int_Df(x,y,z(x,y))\frac{|\nabla F|}{|F_z|}dxdy =\int_Df(x,y,z(x,y))\frac{|\nabla F|}{|F_x|}dydz =\int _Df(x,y,z(x,y))\frac{|\nabla F|}{|F_y|}dzdx  
    \]
    这里将其投影到三个平面上的具体形式都写了出来,进行具体计算时当然选择最容易计算的进行。

    \item 曲面的定向,曲面 $S \subset \mathbb{R}^3$ 可定向(orientable)即能够以连续的方式指定每一个点的法向量,一个定向曲面是指定了单位法向量的可定向曲面。直观上来说,可定向是可以区分曲面的两侧的,不可定向的曲面例子有莫比乌斯带和克莱因瓶。

    \item 第一类曲面积分的产生动机容易理解为计算变密度曲面的质量,自然现象中还有需要计算通过某个面的向量即所谓的通量,这时候被积函数是一个向量,积分微元也是一个向量,设 $S \subset \mathbb{R}^3$是定向曲面, $\mathbf{F}: S \to \mathbf{R}^3$是定义在 $S$ 上的向量场,$\mathbf{F}$在 $S$ 上的第二类曲面积分定义为
    \[
        \int_S\mathbf{F}\cdot d\mathbf{S}:=\int_S\mathbf{F}\cdot\mathbf{n}dS = \int _S \mathbf{F} \cdot d \mathbf{S}
    \]
    其中 $\mathbf{n}$ 是由 $S$ 定向所指定的单位法向量,后面一个是一个记号而已,稍微回顾通量的定义就会发现这就是通量定义的微积分表示。积分值与定向有关,所以
    \[
        \int_{-S}\mathbf{F}\cdot d\mathbf{S}=-\int_S\mathbf{F}\cdot d\mathbf{S}
    \]
    积分微元可以写成
    \[
        d\mathbf{S}=\mathbf{n}dS=(\cos\alpha,\cos\beta,\cos\gamma)dS=(\pm d\sigma_{yz},\pm d\sigma_{zx},\pm d\sigma_{xy})
    \]
    所以
    \begin{align*}
    \int_{S}\mathbf{F}\cdot d\mathbf{S}& =\int_{S}(P\cos\alpha+Q\cos\beta+R\cos\gamma)dS \\
    &=\int_SPdydz+Qdzdx+Rdxdy \\
    &=\int_{D_{yz}}Pdydz+\int_{D_{zx}}Qdzdx+\int_{D_{xy}}Rdxdy.
    \end{align*}
    这里后面两个等式只考虑了当 $\cos \theta _i$取正值也就是 $0 \le \alpha, \beta, \gamma \le \frac{\pi}{2} $的情况,如果它们的值是负值,要相应将加号变成减号。
\end{enumerate}

\subsubsection{题目示例}
\begin{enumerate}
    \item 假设 $S$ 是球面 $x^2 + y^2 + z^2 =a^2$ 落在平面 $z = h(0 < h < a)$上方的部分,计算
    $\int _S \frac{\mathrm{d}S}{z}$

    \item 假设 $S$ 是由平面 $x=0, y=0, z=0$以及 $x+y+z=1$围成的四面体的表面,计算 $\int _S xyz \mathrm{d} S$

    \item 计算 $\int _S xyz \mathrm{d} S$,其中 $S$ 是圆锥面 $z^2 = x^2 + y^2$位于平面 $z=1$和 $z=4$之间的部分。

    \item 计算向量场 $\mathbf{F} = (-y, x, 9)$通过曲面 $S$ 的通量,其中 $S$为定向朝上的球冠,$z=\sqrt{9-x^2-y^2},\quad0\le x^2+y^2\le4$。

    \item 计算 $\int_Sx^2dydz+y^2dzdx+z^2dxdy$,其中 $S$是三维立体 $\Omega = [0,a] \times [0,b] \times [0,c]$的表面,取外侧定向。

    \item 计算 $\int _S xyz d x d y$,其中 $S$ 是球面 $x^2 + y^2 + z^2 = 1, x \ge 0, y \ge 0$取外侧定向。

    \item 计算 $\int _S (z^2 + x)dx dy - z dxdy$,其中 $S$ 是旋转抛物面 $z = \frac{1}{2}(x^2 + y^2)$介于平面 $z=0$和 $z=2$的部分,定向取下侧。
\end{enumerate}

\subsubsection{习题参考}
\begin{enumerate}
    \item (教材第222页习题11-4第6题) 计算下列对面积的曲面积分
        \begin{enumerate}[(1)]
            \item $\int _\Sigma (z + 2x + \frac{4}{3}y) \mathrm{d} S$,其中 $\Sigma$为平面 $\frac{x}{2} + \frac{y}{3} + \frac{z}{4} =1$在第一卦限中的部分。
    
            \item $\int _\Sigma (2xy - 2x^2 -x + z) \mathrm{d}S$,其中 $\Sigma$ 为平面 $2x + 2y + z=6$在第一卦限中的部分。
    
            \item $\int _\Sigma (x+y+z) \mathrm{d}S$,其中 $\Sigma$ 为球面 $x^2 + y^2 + z^2=a^2$上 $z \ge h(0 < h < a)$的部分。
    
            \item $\int _\Sigma (xy + yz + zx)\mathrm{d}S$,其中 $\Sigma$为锥面 $z=\sqrt{x^2 + y^2}$被柱面 $x^2 + y^2 = 2ax$所截得的有限部分。
        \end{enumerate}
    
    \item (教材第223页习题11-4第7题) 求抛物面壳 $z = \frac{1}{2}(x^2 + y^2)(0 \le z \le 1)$的质量,此壳的面密度为 $\mu =z$。
    
    \item (教材第231页习题11-5第3题) 计算下列对坐标的曲面积分:
        \begin{enumerate}[(1)]
            \item $\int _\Sigma x^2y^2z dxdy$,其中 $\Sigma$是球面 $x^2 + y^2 + z^2=R^2$的下半部分的下侧。
            
            \item $\int _\Sigma z dxdy + x dydz + y dzdx$,其中 $\Sigma$是柱面 $x^2 + y^2 =1$被平面 $z=0$及 $z=3$所截得的在第一卦限内的部分的前侧。
    
            \item $\int _\Sigma \left( f(x,y,z) + x\right) dydz + \left( 2f(x,y,z) +y \right) dzdx + \left( f(x,y,z) + z \right) dxdy$,其中 $f(x,y,z)$为连续函数,$\Sigma$是平面 $x-y+z=1$在第四卦限部分的上侧。
    
            \item $\int _\Sigma xz dydz + xydydz + yz dzdx$,其中 $\Sigma$是平面 $x=0,y=0,z=0,x+y+z=1$所围成的空间区域的整个边界曲面的外侧。
        \end{enumerate}
    
    \item (教材第231页习题11-5第4题) 把对坐标的曲面积分
    \[
        \int _\Sigma P(x,y,z) dydz + Q(x,y,z) dzdx + R(x,y,z)dxdy
    \]
    化成对面积的曲面积分,其中
        \begin{enumerate}[(1)]
            \item $\Sigma$ 是平面 $3x + 2y + 2\sqrt{3}z=6$在第一卦限的部分的上侧。
            \item $\Sigma$ 是抛物面 $z =8 - (x^2 + y^2)$在 $xOy$ 面上方的部分的上侧。
        \end{enumerate}
\end{enumerate}

\subsection{高斯公式和斯托克斯公式}
\subsubsection{知识概要}
\begin{enumerate}
    \item 高斯公式:假设三维闭区域 $\Omega \subset \mathbb{R}^3$ 的边界为分片光滑定向曲面 $S$ ,若 $\mathbf{F} = (P, Q, R) \in C^1(\Omega)$,则
    \[
        \iint_SPdydz+Qdzdx+Rdxdy=\iiint_\Omega\left(\frac{\partial P}{\partial x}+\frac{\partial Q}{\partial y}+\frac{\partial R}{\partial z}\right)dxdydz
    \]
    引入散度的,称 $\nabla \cdot \mathbf{F}$为 $\mathbf{F}$ 的散度,上面的公式就可以写成
    \[
        \iint_S\mathbf{F}\cdot\mathbf{n}dS
        = \int _S \mathbf{F}\cdot \mathrm{d} \mathbf{S}
        =\iiint_\Omega\mathbf{div}\mathbf{F}dV = \iiint_\Omega \nabla \cdot  \mathbf{F}dV
    \]
    其中 $\mathbf{n}$是 $S$的单位外法向。高斯公式可以认为是格林公式(向量形式)在三维的推广。

    \item 散度和高斯公式的物理意义。向量场 $\mathbf{F}$ 散度的定义如下
    \[
        \mathbf{div} \mathbf{F}(\mathbf{r}) = \lim _{V \to 0} \frac{\int _{S(V)} \mathbf{F} \cdot \mathrm{d} \mathbf{S}}{V}
    \]
    $V$是包含点 $\mathbf{r}$的一个小体积, $S(V)$是 $V$ 的表面积,取定坐标系可以将其定义式继续化简,最终得到 $\mathbf{div F} = \nabla \cdot \mathbf{F}$,从其定义式中可以看出其物理(几何)意义为向量场在某一点的通量密度,散度大于0对应该处是“源”,貌似有场从该处往外发出,小于0对应该处是“汇”,貌似场汇聚消失在该处,高斯公式的物理意义是向量场通过某个闭合曲面的通量等于该体积内通量密度的积分。在电磁学中计算电场的通量时对这个公式的理解会更加深入。这里取直角坐标系对上面的结论略微推导,感兴趣的同学可以细看,取点 $\mathbf{r} = (x,y,z)$被包括在小六面体 $[x - \frac{\Delta x}{2}, x+ \frac{\Delta x}{2}] \times [y - \frac{\Delta y}{2}, y+ \frac{\Delta y}{2}] \times [z - \frac{\Delta z}{2}, z+ \frac{\Delta z}{2}] $中,小六面体的体积为 $\Delta x \Delta y \Delta z$,场 $\mathbf{F} = (P(\mathbf{r}), Q(\mathbf{r}), R(\mathbf{r}))$,计算定义式中的积分项
    \begin{align*}
        \int _{S(V)} \mathbf{F} \cdot \mathrm{d} \mathbf{S}
        =& \Delta x \Delta y (R + \frac{\partial R}{\partial x} \frac{\Delta x}{2} + \frac{\partial R}{\partial y} \frac{\Delta y}{2} + \frac{\partial R}{\partial z} \frac{\Delta z}{2} ) - \Delta x \Delta y (R + \frac{\partial R}{\partial x} \frac{\Delta x}{2} + \frac{\partial R}{\partial y} \frac{\Delta y}{2} - \frac{\partial R}{\partial z} \frac{\Delta z}{2} ) \\
        +& \Delta y \Delta z \left( P + \nabla P \cdot (\frac{\Delta x}{2},\frac{\Delta y}{2}, \frac{\Delta z}{2} ) -  \Delta y \Delta z (P + \nabla P \cdot (-\frac{\Delta x}{2},\frac{\Delta y}{2}, \frac{\Delta z}{2} ) \right) \\
        +& \Delta z \Delta x \left( P + \nabla P \cdot (\frac{\Delta x}{2},\frac{\Delta y}{2}, \frac{\Delta z}{2} ) -  \Delta z \Delta x (P + \nabla P \cdot (\frac{\Delta x}{2},-\frac{\Delta y}{2}, \frac{\Delta z}{2} ) \right) \\
        =& \Delta x \Delta y \Delta z (\frac{\partial P}{\partial x}+\frac{\partial Q}{\partial y}+\frac{\partial R}{\partial z})
    \end{align*}

    \item 斯托克斯公式是将格林公式从 $\mathbb{R}^2$ 上的曲线推广到了 $\mathbb{R}^3$中的曲线,假设 $\mathbf{F}=(P,Q,R) \in C^1(S)$,则
    \[
        \oint_CPdx+Qdy+Rdz=\iint_S\left(\frac{\partial R}{\partial y}-\frac{\partial Q}{\partial z}\right)dydz+\left(\frac{\partial P}{\partial z}-\frac{\partial R}{\partial x}\right)dzdx+\left(\frac{\partial Q}{\partial x}-\frac{\partial P}{\partial y}\right)dxdy
    \]
    其中 $S \subset \mathbb{R}^3$ 是二维(只需要两个独立自由变量描述)光滑曲面,$C = \partial S$是曲面 $S$ 的边界,并且以右手定则指定 $C$的定向,即右手沿着 $C$的定向握拳时大拇指的方向为 $S$的法向量方向。如果 $S$是在一个平面上时 $(z=0)$,斯托克斯公式就退化成了格林公式。

    \item 向量场 $\mathbf{F}$ 旋度的定义如下
    \[
        \mathbf{curl F} \cdot \hat{n} = \lim _{S \to 0} \frac{\oint _{l(S)} \mathbf{F} \cdot \mathrm{d} \mathbf{l}}{S}
    \]
    其中 $S$是一个趋于0的小面元,其法向(定向)为 $\hat{n}$,$l(S)$是以右手定则规定的绕 $S$的边界回路,选定坐标系后可以化简为
    \[
        \mathrm{curl}\boldsymbol{A}=\left(\frac{\partial A_{z}}{\partial y}-\frac{\partial A_{y}}{\partial z}\right)\boldsymbol{\hat{x}}+\left(\frac{\partial A_{x}}{\partial z}-\frac{\partial A_{z}}{\partial x}\right)\boldsymbol{\hat{y}}+\left(\frac{\partial A_{y}}{\partial x}-\frac{\partial A_{x}}{\partial y}\right)\boldsymbol{\hat{z}}
        = \nabla \times \mathbf{F}
    \]
    记号 $\nabla \times \mathbf{F}$ 非常简练
    \[
    \left.\mathbf{curl F}:=\nabla\times\mathbf{F}=\left|\begin{array}{ccc}\mathbf{i}&\mathbf{j}&\mathbf{k}\\\partial_x&\partial_y&\partial_z\\P&Q&R\end{array}\right.\right|
    \]
    说散度是通量密度的话,旋度可以说是旋量(面)密度;散度可以考察某个点是“源”还是“汇”,旋度是考察某个点的旋转情况,在学习电磁学的过程中,应该会对旋度有更加深入的理解,在深入理解之前可以从形式上记住斯托克斯公式
    \[
        \oint _{l(S)} \mathbf{F} \cdot \mathrm{d} \mathbf{l} = \int _S \nabla \times \mathbf{F} \cdot \mathrm{d} \mathbf{S} = \int _S \nabla \times \mathbf{F} \cdot \hat{n} \mathrm{d} S
    \]
\end{enumerate}

\subsubsection{题目示例}
\begin{enumerate}
    \item 计算 $\int _S (x-y) dx dy + (y-z)x dydz $,其中 $S$ 是圆柱面 $x^2 + y^2 =1$和平面 $z=0, z=3$所围成的边界曲面。

    \item 计算 $\int _S (x^2\cos\alpha+y^2\cos\beta+z^2\cos\gamma)dS$,其中 $S$ 是圆锥面 $x^2 + y^2 =z^2$介于平面 $z=0$和 $z=h$的部分,$\cos \alpha, \cos \beta, \cos \gamma$是 $S$的朝下的法向量的方向余弦。
    
    \item 假设 $S$是三维闭区域 $\Omega \subset \mathbf{R}^3$的边界曲面,$\mathbf{n}$是其单位外法相,$u,v\in C^2(\Omega)$是定义在 $\Omega$上的函数,证明
    \[
        \int_\Omega u\Delta vdV=\int_Su\frac{\partial v}{\partial\mathbf{n}}dS-\int_\Omega\nabla u\cdot\nabla vdV
    \]

    \item 计算 $\oint _C zdx+xdy+ydz$,其中 $C$ 是平面 $x+y+z=1$与坐标平面相交所得三角形,从上方看取逆时针。

    \item 假设 $\Omega \in \mathbb{R}^3$是一个单连通的三维区域,$\mathbf{F} = (P,Q,R) \in C^1(\Omega)$,证明 $\mathbf{F}$是保守场当且仅当 $\mathbf{curl F} = 0$ (提示:$\mathbf{F}$是保守场指存在势函数 $f$ 使得 $\mathbf{F} = \nabla f$,这等价于 $\int _C \mathbf{F} \cdot \mathrm{d} \mathbf{r}$与路径无关)。
\end{enumerate}

\subsubsection{习题参考}
\begin{enumerate}
    \item (教材第239页习题11-6第1题) 利用高斯公式计算曲面积分
        \begin{enumerate}[(1)]
            \item $\oint _\Sigma x^2dydz + y^2dzdx + z^2 dxdy$,其中 $\Sigma$ 为平面 $x=0,y=0,z=0,x=a,y=a,z=a$所围成的立体的表面的外侧。
    
            \item $\oint _\Sigma x^3 dydz + y^3 dzdx + z^3 dxdy$,其中 $\Sigma$ 为球面 $x^2 + y^2 + z^2 =a^2$的上半侧。
    
            \item $\oint xz^2 dydz + (x^2y - z^3)dzdx + (2xy + y^2z)dxdy$,其中 $\Sigma $为上半球体 $0 \le z \le \sqrt{a^2 -x^2 -y^2}$,$x^2 + y^2 \le a^2$的表面的外侧。
    
            \item $\oint xdydz + ydzdx + zdxdy$,其中 $\Sigma$是界于 $z=0$ 和 $z=3$ 之间的圆柱体 $x^2 + y^2 \le 9$的整个表面外侧。
    
            \item $\oint _\Sigma 4xzdydz - y^2dzdx + yz dxdy$,其中 $\Sigma$ 是平面 $x=0,y=0,z=0,x=1,y=1,z=1$所围成立方体的全表面外侧。
        \end{enumerate}

    \item (教材第240页习题11-6第5题) 利用高斯公式推证阿基米德原理:浸没在液体中的物体所受液体的压力和合力(即浮力)的方向铅直向上,其大小等于这物体所排开的液体的重力。

    \item (教材第248页习题11-7第2题) 利用斯托克斯公式,计算下列曲线积分
        \begin{enumerate}[(1)]
            \item $\int _\Gamma y dx + zdy + xdz$,其中 $\Gamma$为圆周 $x^2 + y^2 + z^2 = a^2, x + y + z =0$,若从 $x$轴的正向看去,这圆周是取逆时针方向。
    
            \item $\oint _\Gamma (y-z)dx + (z-x)dy + (x-y)dz$,其中 $\Gamma$为椭圆 $x^2 + y^2 =a^2, \frac{x}{a} + \frac{z}{b} =1 (a>0, b>0)$,若从 $x$ 轴正向看去,这椭圆是取逆时针方向。
    
            \item $\oint _\Gamma 3ydx -xzdy + yz^2dz$,其中 $\Gamma$是圆周 $x^2 + y^2 =2z, z=2$,若从 $z$正向看去,这圆周是取逆时针方向。
    
            \item $\oint _\Gamma 2ydx + 3xdy -z^2dz$,其中 $\Gamma$ 是圆周 $x^2 + y^2 + z^2 =9,z=0$,若从 $z$正向看去,这圆周是取逆时针方向。
        \end{enumerate}

    \item (教材第249页习题11-7第7题) 设 $u = u(x,y,z)$具有二阶连续偏导数,求 $\mathbf{rot(grad )}u$
\end{enumerate}
\section{题目答案参考}
\subsection{欧式空间与多元函数}
\subsubsection{题目示例}
\begin{enumerate}
\item 求函数 $f(x)$ 在原点的极限
$$
f(x,y)=(x^2+y^2)\sin\frac1{x^2+y^2},\quad(x,y)\neq(0,0)
$$
解答:
\begin{align*}
    &\lim_{(x,y) \to (0,0)} f(x,y) = 0 \\
    &\forall \varepsilon > 0 , \exists \delta = \sqrt{\varepsilon} , \text{s.t.} \forall 0<|\mathbf{x}-0|=\sqrt{(x-0)^2 + (y-0)^2}<\delta, |f(x,y)-0|\le |(x^2+y^2)|< \varepsilon
\end{align*}


\item 计算
$$
\lim_{(x,y)\to(0,2)}\frac{\sin(xy)}x
$$
解答:
\begin{align*}
     \lim_{(x,y)\to(0,2)}\frac{\sin(xy)}x = \lim_{(x,y)\to(0,2)}\frac{\sin(xy)}{xy} y = 2
\end{align*}
这个理解就好了,就不要太追求太严格的证明了。哈哈哈。

\item 证明函数 $f(x)$ 在原点的极限不存在
$$
f(x,y)=\frac{xy}{x^2+y^2},\quad(x,y)\neq(0,0)
$$
解答
\begin{align*}
    \lim_{x\to 0} \lim_{y \to 0} f(x,y) = 0, \lim_{x \to 0}\lim_{y \to x} = \frac{1}{2}
\end{align*}
不同的累次极限不相等,所以在原点的极限(重极限)不存在。
\end{enumerate}
\subsubsection{习题参考}
\begin{enumerate}
\item 用数学语言写出下列概念的严格定义:内点、外点、边界点、孤立点、聚点、开集、闭集。
解答:参照这一章节的知识概要,那里已经用数学语言写出来了,也就是用 $\varepsilon - \delta$语言写了,$\varepsilon - \delta$语言是个好语言,非常凝练,可以常常试着应用,非常精妙的(但是偶尔可能比较繁琐)。

\item 找出集合 $S=\left\{\left(\frac1m,\frac1n\right)\in\mathbb{R}^2|m,n\in\mathbb{N}\right\}$ 的边界点和聚点。
解答:在平面上画出这个点集的大概形状(除开有理数外还有大量的无理数),根据边界点的定义:任何一个邻域内,既有属于集合内的,又有不属于集合内的。故集合内所有点都是边界点,再关注以下特殊的点,容易判断,总的边界点是
\begin{align*}
    S \cup \left\{ (0,0), \left(\frac{1}{m},0 \right),\left(0, \frac{1}{n}\right) |m,n\in\mathbb{N} \right\}
\end{align*}

根据聚点的定义:在任何一个去心邻域内,一定有属于集合内的,所以聚点是
\begin{align*}
    \left\{ (0,0), \left(\frac{1}{m},0 \right),\left(0, \frac{1}{n}\right) |m,n\in\mathbb{N} \right\}
\end{align*}

\item 利用极限的定义证明 $m$ 维点列极限的以下性质:
    \begin{enumerate}[(1)]
    
    \item $\lim _{n\to \infty} \mathbf{a}_n = \mathbf{A}$当且仅当点 $\mathbf{a}_n$的每个位置的分量构成的数列收敛到点 $\mathbf{A}$ 对应分量

    解答:
    \begin{align*}
        & \lim _{n\to \infty} \mathbf{a}_n = \mathbf{A} \Leftrightarrow \forall \varepsilon>0,\exists N \in \mathbb{N},\text{ s.t. }|\mathbf{a}_n - \mathbf{A}|< \varepsilon, \forall n > N \\
        & \text{对于任意一个分量,都有}  |a_i - A_i| \le |\mathbf{a}_n - \mathbf{A}| < \varepsilon 
    \end{align*}
    所以每个分量都收敛到对应的分量。
    
    \item 若$\lim _{n\to \infty} \mathbf{a}_n = \mathbf{A}$则$\{ \mathbf{a}_n \}$有界
    解答:
    \begin{align*}
        & \lim _{n\to \infty} \mathbf{a}_n = \mathbf{A} \Leftrightarrow \forall \varepsilon>0,\exists N \in \mathbb{N},\text{ s.t. }|\mathbf{a}_n - \mathbf{A}|< \varepsilon, \forall n > N \\
        & 1 \le n \le N, |\mathbf{a}_n| \le \max(\mathbf{a}_1, \mathbf{a}_2, \cdots, \mathbf{a}_n), \text{对于} n > N , \text{取} \varepsilon=1, \text{有} |\mathbf{a}_n| < |A|+1
    \end{align*}
    所以有界 $|\mathbf{a}_n| \ge \max(a_1, a_2, \cdots ,a_n, |\mathbf{A}|+1)$ 。上面用到了三角不等式,两边之和大于第三边 $|a|+|b|>|a \pm b|$,两边之差小于第三边 $|a|-|b| < |a \pm b|$。
    
    \item $\lim _{n\to \infty} \mathbf{a}_n \cdot \mathbf{b}_n = \lim _{n\to \infty} \mathbf{a}_n \cdot \lim _{n\to \infty} \mathbf{b}_n $
    解答:
    根据证明1,每个位置的分量收敛到对应的分量,再由于一元微积分证明的 $\lim_{n\to \infty} (x_n y_n) = \lim_{n\to \infty} x_n \lim_{n\to \infty} y_n  $得证。
    \end{enumerate}

\end{enumerate}

\subsection{偏微分}
\subsubsection{题目示例}
\begin{enumerate}
\item 曲面$z=f(x,y)=\sqrt{9-2x^2-y^2}$与平面$y=1$交于一条曲线$\gamma.$试求曲线
$\gamma$在$(\sqrt2,1,2)$点的切线。
解答:
\begin{equation*}
\left. \frac{\partial f}{\partial x}  \right|_{x_0}  = \frac{-2x}{\sqrt{9-2x^2-y^2}} = - \sqrt{2}
\end{equation*}
所以切线方程是
\begin{equation*}
    z-2 = -\sqrt{2}(x-\sqrt{2}), y=1
\end{equation*}


\item 求函数 $f(x)$ 在 $(0,0)$的偏导数
$$
f(x,y)=\begin{cases}\frac{xy}{x^2+y^2},&(x,y)\neq(0,0)\\0,&(x,y)=(0,0) \end{cases}
$$
解答:
\begin{align*}
    \frac{\partial f }{\partial x} = \frac{y^3 - x^2y}{(x^2+y^2)^2} \quad
    \frac{\partial f }{\partial y} = \frac{x^3 - y^2x}{(x^2+y^2)^2}
\end{align*}

\end{enumerate}
\subsubsection{习题参考}
\begin{enumerate}
\item 证明函数
$$
u(x)=\frac1{|x|}:\mathbb{R}^3\setminus\{0\}\to\mathbb{R}
$$
满足方程 $\Delta u $是我们新认识的一个记号。(求模的 $x$应该粗写,但是从上下文看不会造成歧义,所以没有粗写,我们之前已经做过约定)
$$
\Delta u=\frac{\partial^2u}{\partial x^2}+\frac{\partial^2u}{\partial y^2}+\frac{\partial^2u}{\partial z^2}=0.
$$
解答:$u(x) = \frac{1}{\sqrt{x^2 + y^2 + z^2}}$
\begin{equation*}
    \frac{\partial^2 u}{\partial x^2} = \frac{\partial }{\partial x} \frac{\partial u}{\partial x} = \frac{\partial }{\partial x} \frac{-x}{(x^2 + y ^2 + z ^2)^{\frac{3}{2}}} = \frac{2x^2-y^2-z^2 }{(x^2+y^2+z^2)^\frac{5}{2}}
\end{equation*}
同理容易求出另外两个偏导数,加在一起等于0.

\item 设 $e_i$是沿 $x_i$ 方向的单位向量,证明 $f$ 在 $x$ 点对 $x_i$ 的偏导数存在的充要条件是: $f$ 在 $x$ 点沿 $e_i$ 和 $-e_i$ 的方向导数存在且互为相反数。

证明:先证明必要条件, $f$ 在 $x$ 点 对$x_i$ 的偏导数存在,即下面这个极限存在,我们将其记为 $A$
\begin{equation*}
    \lim _{\Delta t \to 0} \frac{f(\cdots, x_i + \Delta t,\cdots)-f(x)}{\Delta t} = A, \quad f(\cdots, x_i + \Delta t,\cdots) = f(x) + A \Delta t
\end{equation*}
从而,根据方向导数的定义
\begin{equation*}
   \frac{\partial f}{\partial (e_i)} = \lim _{\Delta t \to 0 ^+ } \frac{f(\cdots, x_i + \Delta t,\cdots)-f(x)}{\Delta t} =\lim _{\Delta t \to 0^+} \frac{A \Delta t}{\Delta t} =  A
\end{equation*}
而另外一个偏导数
\begin{equation*}
    \frac{\partial f}{\partial (-e_i)} = \lim _{\Delta t \to 0 ^+ } \frac{f(\cdots, x_i +( - \Delta t),\cdots)-f(x)}{\Delta t} = \lim _{\Delta t \to 0^+} \frac{-A \Delta t}{\Delta t} = -A
\end{equation*}
所以,$f$ 在 $x$ 点沿 $e_i$ 和 $-e_i$的方向导数存在且互为相反数。再看充分条件,即如下极限存在,我们假设为 $A$ 和 $B$
\begin{align*}
    \lim _{\Delta t \to 0 ^+ } \frac{f(\cdots, x_i +\Delta t,\cdots)-f(x)}{\Delta t} = A, \quad f(\cdots,x_i+\Delta t , \cdots) = f(x) + A \Delta t , \Delta t \to 0^+ \\
    \lim _{\Delta t \to 0 ^+ } \frac{f(\cdots, x_i +(- \Delta t),\cdots)-f(x)}{\Delta t} = B , \quad  f(\cdots,x_i+\Delta t , \cdots) = f(x) + B \Delta t , \Delta t \to 0^-
\end{align*}
如果 $A$ 和 $B$ 互为相反数,则有
\begin{equation*}
    f(\cdots,x_i+\Delta t , \cdots) = f(x) + A \Delta t
\end{equation*}
从而 $f$ 在点 $x$ 对 $x_i$ 的偏导数存在。(这道题有点在考偏导数,方向导数的定义和概念,会用数学语言表达这些概念,做起来会容易一些)

\item (教材第71页习题9-2第1题) 求下列函数的偏导数
\begin{align*}
(1). z &= x^3y-y^3x  & (2). s &= \frac{u^2+v^2}{u v} \\
(3). z &= \sqrt{\ln(x y)} &   (4). z &= \sin(x y)  + \cos ^2(x y) \\
(5). z &= \ln(\tan(\frac{x}{y})) & (6). z &= (1+x y)^y \\
(7). u &= x^{\frac{y}{z}} & 8. u &= \arctan (x-y)^z 
\end{align*}
解:(求偏导的时候就是将不是求导的变量看成常数)
\begin{align*}
    (1). \frac{\partial z }{\partial x} = 3x^2y - y^3, & \frac{\partial z }{\partial y} = x^3 - 3y^2 x \\
    (2). \frac{\partial s}{\partial u} = &  \frac{\partial s}{\partial w} \\
    (3). \frac{\partial z}{\partial x} & \frac{\partial z}{\partial y} \\
    (4). \frac{\partial z}{\partial x} & \frac{\partial z}{\partial y} \\
    (5). \frac{\partial z}{\partial x} & \frac{\partial z}{\partial y} \\
    (6). \frac{\partial z}{\partial x} & \frac{\partial z}{\partial y} \\
    (7). \frac{\partial z}{\partial x} & \frac{\partial z}{\partial y} \\
    (8). \frac{\partial z}{\partial x} & \frac{\partial z}{\partial y}
\end{align*}


\item (教材第71页习题9-2第5题)曲线 $L$ 在点 $(2,4,5)$ 处的对于 $x$ 轴的倾角为多少?
$$
L: \left\{\begin{matrix}
z = \frac{x^2+y^2}{4}\\
y = 4
\end{matrix}\right.
$$
解:根据题意和偏导数的意义,求偏导数
\begin{equation*}
    \frac{\partial z}{\partial x} = \frac{1}{2} x
\end{equation*}
代入 $(2,4,5)$ 得到 $ \left. \frac{\partial z}{\partial x} \right|_{x=2} = 1$,写出切线方程
\begin{equation*}
    z = x + 3
\end{equation*}
所以对 $x$ 轴的倾角为 $45 ^ \circ$。

\item (教材第71页习题9-2第7题)设 $f(x,y,z) = x y^2+y z^2+z x^2$,求 $f_{xx}(0,0,1)$, $f_{xz}(1,0,2)$, $f_{yz}(0,-1,0)$, $f_{zzx}(2,0,1)$

解:还是一道求偏导的题目,直接求偏导并代入对应的值即可
\begin{align*}
    f_{xx} = 2z , \quad f_{xx}(0,0,1) = 0 \\
    f_{xz} = 2x , \quad f_{xz}(1,0,2) = 2 \\
    f_{yz} = 2z , \quad f_{yz}(0,-1,0) = 0 \\
    f_{zzx} = 0 , \quad f_{zzx}(2,0,1) = 0
\end{align*}
\end{enumerate}

\subsection{全微分}
\subsubsection{习题参考}
\begin{enumerate}
    \item 证明函数 $u(x)=|x|:\mathbb{R} ^n \to \mathbb{R}$满足
    $$
    |\nabla u(x) | = 1, \forall x \in \mathbb{R}^n
    $$
    解:根据偏导的定义
    
    \item (教材第77页习题9-3第1题)求下列函数的全微分
    \begin{align*}
    (1). z & = x y + \frac{x}{y}; &  (2). z &= \mathrm{e}^{\frac{y}{x}}; \\
    (3). z &= \frac{y}{x^2 + y^2}; & (4). u & = x^{yz}
    \end{align*}
    
    \item (教材第77页习题9-3第2题)求函数 $z = \ln (1+x^2 + y^2)$当 $x=1,y=2$时的全微分。
    
    \item (教材第77页习题9-3第3题)求函数 $z = \frac{y}{x}$当 $x=2,y=1, \Delta x = 0.1, \Delta y = -0.2$时的全增量和全微分。
\end{enumerate}

\end{document}